\section{Построение кривой Безье по набору точек}

Прежде всего стоит отметить, что именно мы будем иметь в виду под кривой Безье. Дело в том, что в классическом
понимании у кривой Безье только первая производная непрерывна, а в наших задачах получаемая кривая имеет непрерывные
производные до произвольного порядка. Поэтому можно ввести дополнительное понятие \textit{улучшенной кривой Безье},
но для краткости мы будем называть её просто кривой Безье.

Характерными особенностями такой кривой является способ её построения и её внешний вид. В каждой из последующих задач
способ построения описан довольно подробно.

\subsection*{На плоскости}

Рассмотрим плоскость действительных чисел $\mathbb{R}^2$. Пусть на ней задана последовательность различных точек $p_i$:

$${p_i: (x_i, y_i)}, i \in {1, \dots, k}.$$

Будем предполагать, что в последовательности присутствует более двух точек, т. е. $k > 2$.

Пусть каждые две последовательные точки $p_i$ и $p_{i+1}$ соединены отрезками, и эти отрезки поделены пополам точками
$q_i, i \in {1, \dots, k-1}$.

Требуется построить параметризованную кривую $r(t) \in \mathbb{R}^2$, проходящую последовательно через точки $p_1,
q_1, \dots, q_{k-1}, p_k$ в направлении построенных отрезков. При этом данная кривая должна принадлежать классу
$C^n$, т.~е. иметь непрерывные производные в каждой точке до порядка $n$ включительно.

\subsection*{На двумерной сфере}

TODO

\subsection*{На ориентационной сфере}

TODO
