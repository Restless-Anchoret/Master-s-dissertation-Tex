\section{Интерполирование кривой по набору точек и направлениям касательных}

В каждой из задач данного раздела предполагается, что $k>3$. Также имеется в виду, что углы $\phi_i$, вообще говоря,
в некоторых точках могут быть не заданы.

\subsection*{На плоскости}

Рассмотрим плоскость действительных чисел $\mathbb{R}^2$. Пусть на ней задана последовательность различных точек
$p_i$, а также задана последовательность углов $\phi_i$:

$${p_i: (x_i, y_i)}, {\phi_i}, i \in {1, \dots, k}.$$

Требуется построить параметризованную кривую $r(t) \in \mathbb{R}^2$, проходящую последовательно через все точки
$p_i$, причём в $i$-й точке "--- под углом $\phi_i$ к оси Ox (если угол в данной точке задан). При этом данная кривая
должна принадлежать классу $C^n$, т. е. иметь непрерывные производные в каждой точке до порядка $n$ включительно.

\subsection*{На двумерной сфере}

Рассмотрим двумерную сферу $S^2$, заданную в некоторой ортогональной системе координат с началом в центре этой сферы.
Пусть задана последовательность различных точек $p_i$, лежащих на поверхности этой сферы, а также задана
последовательность углов $\phi_i$:

$${p_i}, {\phi_i}, i \in {1, \dots, k}.$$

Требуется построить параметризованную кривую $r(t)$, лежащую на поверхности сферы $S^2$ и проходящую последовательно
через все точки $p_i$, причём в $i$-й точке "--- под углом $\phi_i$ к экватору сферы $S^2$ (если угол в данной точке
задан). При этом данная кривая должна принадлежать классу $C^n$, т. е. иметь непрерывные производные в каждой точке
до порядка $n$ включительно.

\subsection*{На ориентационной сфере}

Рассмотрим ориентационную сферу $S^3$, заданную в некоторой ортогональной сис\-теме координат с началом в центре этой
сферы. Пусть задана последовательность различных ориентаций $Q_i$, принадлежащих поверхности этой сферы, а также задана
последовательность углов $\phi_i$:

$${Q_i}, {\phi_i}, i \in {1, \dots, k}.$$

Требуется построить параметризованную кривую $R(t)$, лежащую на поверхности сферы $S^3$ и проходящую
последовательно через все ориентации $Q_i$, причём в $i$-й точке "--- под углом $\phi_i$ к экватору сферы $S^3$ (если
угол в данной точке задан). При этом данная кривая должна принадлежать классу $C^n$, т. е. иметь непрерывные
производные в каждой точке до порядка $n$ включительно.
