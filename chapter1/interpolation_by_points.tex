\section{Интерполирование кривой по набору точек}

\subsection*{На плоскости}

Рассмотрим плоскость действительных чисел $\mathbb{R}^2$. Пусть на ней задана последовательность различных точек $p_i$:

$${p_i: (x_i, y_i)}, i \in {1, \dots, k}.$$

Будем предполагать, что в последовательности присутствует более трёх точек, т. е. $k > 3$.

Требуется построить параметризованную кривую $r(t) \in \mathbb{R}^2$, проходящую последовательно через все точки
$p_i$. При этом данная кривая должна принадлежать классу $C^n$, т. е. иметь непрерывные производные в каждой точке до
порядка $n$ включительно.

\subsection*{На двумерной сфере}

Рассмотрим двумерную сферу $S^2$, заданную в некоторой ортогональной системе координат с началом в центре этой сферы.
Пусть задана последовательность различных точек $P_i$, лежащих на поверхности этой сферы:

$${P_i}, i \in {1, \dots, k}.$$

Будем предполагать, что в последовательности присутствует более трёх точек, т. е. $k > 3$.

Требуется построить параметризованную кривую $r(t)$, лежащую на поверхности сферы $S^2$ и проходящую последовательно
через все точки $P_i$. При этом данная кривая должна принадлежать классу $C^n$, т. е. иметь непрерывные производные в
каждой точке до порядка $n$ включительно.

\subsection*{На ориентационной сфере}

Рассмотрим ориентационную сферу $S^3$, заданную в некоторой ортогональной сис\-теме координат с началом в центре этой
сферы. Пусть задана последовательность различных ориентаций $Q_i$, принадлежащих поверхности этой сферы:

$${Q_i}, i \in {1, \dots, k}.$$

Будем предполагать, что в последовательности присутствует более трёх ориентаций, т. е. $k > 3$.

Требуется построить параметризованную кривую $R(t)$, лежащую на поверхности сферы $S^3$ и проходящую
последовательно через все ориентации $Q_i$. При этом данная кривая должна принадлежать классу $C^n$, т. е. иметь
непрерывные производные в каждой точке до порядка $n$ включительно.
