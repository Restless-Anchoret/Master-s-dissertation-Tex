\section{На плоскости}

Рассмотрим задачу построения кривой Безье на плоскости. Дан набор $k$ точек $\{p_i\}$ в $\mathbb{R}^2$.
Нужно построить непрерывно-дифференци\-руемую до заданного порядка кривую Безье на основе данного набора точек.

\subsection*{Построение отрезка между двумя точками}

Рассмотрим вначале некоторые две различные точки $p_1$ и $p_2$ в $\mathbb{R}^2$. Отрезок, соединяющий эти две точки,
можно выразить следующей формулой:

\begin{equation}
p(u)=(1-u)p_1+up_2, \quad u \in [0,1].
\label{plane-segment}
\end{equation}

Кроме того, точка, лежащая на середине этого отрезка, может быть найдена по формуле:

\begin{equation}
q=\frac{1}{2}(p_1+p_2).
\label{plane-segment-mid}
\end{equation}

Также две половины этого отрезка, сходящиеся, соответственно, от точек $p_1$ и $p_2$ к точке $q$ выражаются формулами:

\begin{equation}
s(u)=(1-\frac{u}{2})p_1+\frac{u}{2}p_2, \quad v(u)=\frac{u}{2}p_1+(1-\frac{u}{2})p_2, \quad u \in [0,1].
\label{plane-segment-halfs}
\end{equation}

Длину отрезка можно вычислить по следующей формуле:

\begin{equation}
L=||p_2-p_1||=\sqrt{(p_2-p_1)_x^2+(p_2-p_1)_y^2}.
\label{plane-segment-length}
\end{equation}

\subsection*{Сглаживание отрезков}

Теперь рассмотрим три различные точки $p_1$, $p_2$ и $p_3$, соединённые последовательно двумя отрезками $p(u)$ и $q(u)$:

$$
p(u)=(1-u)p_2+up_1, \quad q(u)=(1-u)p_2+up_3, \quad u \in [0,1].
$$

Таким образом, кривые имеют следующие граничные точки:

$$
p(0)=q(0)=p_2, \quad p(1)=p_1, \quad q(1)=p_3.
$$

Требуется построить новую кривую $r(u)$, удовлетворяющую следующим граничным условиям:

$$
r^{(k)}(0)=p^{(k)}(1), \quad r^{(k)}(1)=q^{(k)}(1),
$$

$$
k \in {0,1,\dots,l}, \quad l \in \mathbb{N}.
$$

Применим теорему о сглаживании кривых. Векторы из $\mathbb{R}^2$, из которых состоят отрезки $p(u)$ и $q(u)$,
принадлежат мультипликативной группе с операцией сложения векторов. Имеем две кривые в пространстве $\mathbb{R}^2$:

$$
\tilde p(u)=p(u)-p_2, \quad \tilde q(u)=q(u)-p_2, \quad u \in [0,1].
$$

При $u=0$ они обращаются в ноль, поэтому эти кривые выходят из единицы мультипликативной группы. Значит, к ним можно
применить теорему о сглаживании кривых.

Таким образом, можно показать, что искомая кривая $r(u)$ представима следующим образом:

$$
r(u)=\tilde q(w_l(u)u)+\tilde p((1-w_l(u))(u-1))+p_2, \quad u \in [0,1].
$$

\subsection*{Алгоритм построения кривой}

Приведём теперь детальное описание алгоритма построения кривой Безье $r(t)$ на основе точек $p_1,\dots,p_k$.

\bigskip
\textit{Шаг} 0.

\begin{enumerate}
\item Построим отрезки $p_i(u)$ между каждой парой последовательных точек $p_i$ и $p_{i+1}$ для
$i \in \{1,\dots,k-1\}$, используя формулу (\ref{plane-segment}).
\item Определим точки $q_i$, $i \in \{1,\dots,k-1\}$, как середины отрезков $p_i(u)$, используя формулу
(\ref{plane-segment-mid}).
\item Определим половины отрезков $s_i(u)$ и $v_i(u)$, $i \in \{1,\dots,k-1\}$ при помощи формулы
(\ref{plane-segment-halfs}).
\item Определим также отрезки $\tilde s_i(u)$ и $\tilde v_i(u)$ с помощью следующей формулы:
$$
\tilde s_i(u)=s_i(u)-p_i, \quad \tilde v_i(u)=v_i(u)-p_{i+1}, \quad i \in \{1,\dots,k-1\}.
$$
\end{enumerate}

\bigskip
\textit{Шаг} 1.

Определим первый сегмент $r_1(t)$ кривой $r(t)$ по следующей формуле:

$$
r_1(t)=s_1(u(t)),
$$
\noindent где
$$
u(t)=\frac{t-t_1}{t_2-t_1}, \quad t \in [t_1,t_2].
$$

Значения $t_1$ и $t_2$ следует выбирать, исходя из каких-либо дополнительных кинематических требований или ограничений.

\bigskip
\textit{Шаг} $i$, где $i \in \{2,3,\dots,k-1\}$.

Вычислим сегмент $r_i(t)$ кривой $r(t)$ по следующей формуле:
$$
r_i(t)=\tilde s_i(w_l(u(t))u(t))+\tilde v_{i-1}((1-w_l(u(t)))(1-u(t)))+p_i,
$$
\noindent где
$$
u(t)=\frac{t-t_i}{t_{i+1}-t_i}, \quad t \in [t_i,t_{i+1}].
$$

Здесь $w_l(t)$ "--- сглаживающие полиномы, и $l$ "--- требуемая степень непрерывности кривой $r(t)$. Чтобы обеспечить
параметрическую непрерывность кривой $r(t)$ в точках $q_i$, значения $t_i$ следует выбирать следующим образом:

$$
t_{i+1}=t_i+(t_i-t_{i-1}), \quad i \in \{2,3,\dots,k-1\}.
$$

\bigskip
\textit{Шаг} $k$.

Вычислим последний сегмент $r_k(t)$ кривой $r(t)$ по следующей формуле:

$$
r_k(t)=v_{k-1}(1-u(t)),
$$

\noindent где

$$
u(t)=\frac{t-t_k}{t_{k+1}-t_k}, \quad t \in [t_k,t_{k+1}].
$$

Чтобы обеспечить параметрическую непрерывность кривой $r(t)$ в точке $q_{k-1}$, значение $t_{k+1}$ следует выбирать
следующим образом:

$$
t_{k+1}=t_k+(t_k-t_{k-1}).
$$
