\section{На ориентационной сфере}

Рассмотрим аналогичную задачу на ориентационной сфере. Дан набор $k$ ориентаций $\{Q_i\}$, лежащих на поверхности сферы
единичных кватернионов $S^3$. Нужно построить непрерывно-дифференцируемую до заданного порядка кривую Безье, лежащую
на поверхности $S^3$, на основе данного набора ориентаций.

\subsection*{Построение большой дуги}

Рассмотрим вначале две различные ориентации $Q_1$ и $Q_2$, лежащие на поверхности $S^3$, и поговорим о том, как
построить большую дугу, соединяющую эти две ориентации.

Для этого выразим кватернион $R$, обозначающий поворот из ориентации $Q_1$ в ориентацию $Q_2$:

$$
R=Q_2Q_1^*.
$$

Кватернион $R$ обозначает поворот вокруг оси $n$ на угол $\varphi$:

$$
R=R(n,\varphi).
$$

Тогда большую дугу, соединяющую $Q_1$ и $Q_2$, можно выразить следующей формулой:

\begin{equation}
P(u)=R(n,u\varphi)Q_1, \quad u \in [0,1].
\label{orientation-big-arc}
\end{equation}

Кроме того, ориентация, лежащая на середине этой дуги, может быть найдена по формуле:

\begin{equation}
N=R(n,\frac{\varphi}{2})Q_1.
\label{orientation-big-arc-mid}
\end{equation}

Также две половины большой дуги, сходящиеся, соответственно, от ориентаций $Q_1$ и $Q_2$ к ориентации $N$
выражаются формулами:

\begin{equation}
S(u)=R(n,\frac{u}{2}\varphi)Q_1, \quad V(u)=R(n,(1-\frac{u}{2})\varphi)Q_1, \quad u \in [0,1].
\label{orientation-big-arc-halfs}
\end{equation}

\subsection*{Сглаживание больших дуг}

Теперь рассмотрим три различные ориентации $Q_1$, $Q_2$ и $Q_3$, соединённые последовательно двумя большими
дугами. Пусть первая дуга $P(u)$ будет результатом поворота от ориентации $Q_2$ вокруг оси $m$ на угол $\varphi$ до
ориентации $Q_1$, а вторая дуга $Q(u)$ "--- результатом поворота от ориентации $Q_2$ вокруг оси $n$ на угол $\psi$ до
ориентации $Q_3$:

$$
P(u)=R(m,u\varphi)Q_2, \quad Q(u)=R(n,u\psi)Q_2, \quad u \in [0,1].
$$

Таким образом, кривые имеют следующие граничные точки:

$$
P(0)=Q(0)=Q_2, \quad P(1)=Q_1, \quad Q(1)=Q_3.
$$

Требуется построить новую кривую $R(u)$, удовлетворяющую следующим граничным условиям:

$$
R^{(k)}(0)=P^{(k)}(1), \quad R^{(k)}(1)=Q^{(k)}(1),
$$

$$
k \in {0,1,\dots,l}, \quad l \in \mathbb{N}.
$$

Применим теорему о сглаживании кривых. Кватернионы, представляющие повороты от ориентации $Q_2$, принадлежат
мультипликативной группе с операцией умножения кватернионов. Имеем две кривые в пространстве кватернионов:
$R(m,u\varphi)$ и $R(n,u\psi)$. При $u=0$ они равняются единице, поэтому эти кривые выходят из единицы мультипликативной
группы. Значит, к ним можно применить теорему о сглаживании кривых.

Таким образом, можно показать, что искомая кривая $R(u)$ представима следующим образом:

$$
R(u)=R(n,\omega_l(u)u\psi)R(m,(1-\omega_l(u))(1-u)\varphi)Q_2, \quad u \in [0,1].
$$

\subsection*{Алгоритм построения кривой}

Приведём теперь детальное описание алгоритма построения кривой Безье $R(t)$ на основе ориентаций $Q_1,\dots,Q_k$.

\bigskip
\textit{Шаг} 0.

\begin{enumerate}
\item Вычислим оси $n_i$, центральные углы $\varphi_i$ и большие дуги $P_i(u)$ для каждой пары последовательных ориентаций
$Q_i$ и $Q_{i+1}$, $i \in \{1,\dots,k-1\}$, используя формулу (\ref{orientation-big-arc}).
\item Определим ориентации $N_i$, $i \in \{1,\dots,k-1\}$, как середины больших дуг $P_i(u)$, используя формулу
(\ref{orientation-big-arc-mid}).
\item Определим половины больших дуг $S_i(u)$ и $V_i(u)$, $i \in \{1,\dots,k-1\}$ при помощи формулы
(\ref{orientation-big-arc-halfs}).
\end{enumerate}

\bigskip
\textit{Шаг} 1.

Определим первый сегмент $R_1(t)$ кривой $R(t)$ по следующей формуле:

$$
R_1(t)=S_1(u(t)),
$$
\noindent где
$$
u(t)=\frac{t-t_1}{t_2-t_1}, \quad t \in [t_1,t_2].
$$

Значения $t_1$ и $t_2$ следует выбирать, исходя из каких-либо дополнительных кинематических требований или ограничений.

\bigskip
\textit{Шаг} $i$, где $i \in \{2,3,\dots,k-1\}$.

Вычислим сегмент $R_i(t)$ кривой $R(t)$ по следующей формуле:
$$
R_i(t)=R(n_i,\omega_l(u(t))u(t)\frac{\varphi_i}{2})R(n_{i-1},(1-\omega_l(u(t)))(1-u(t))(-\frac{\varphi_{i-1}}{2}))Q_i,
$$
\noindent где
$$
u(t)=\frac{t-t_i}{t_{i+1}-t_i}, \quad t \in [t_i,t_{i+1}].
$$

Здесь $\omega_l(t)$ "--- сглаживающие полиномы, и $l$ "--- требуемая степень непрерывности кривой $R(t)$. Чтобы обеспечить
параметрическую непрерывность кривой $R(t)$ в ориентациях $N_i$, значения $t_i$ следует выбирать следующим образом:

$$
t_{i+1}=t_i+(t_i-t_{i-1}), \quad i \in \{2,3,\dots,k-1\}.
$$

\bigskip
\textit{Шаг} $k$.

Вычислим последний сегмент $R_k(t)$ кривой $R(t)$ по следующей формуле:

$$
R_k(t)=V_{k-1}(1-u(t)),
$$

\noindent где

$$
u(t)=\frac{t-t_k}{t_{k+1}-t_k}, \quad t \in [t_k,t_{k+1}].
$$

Чтобы обеспечить параметрическую непрерывность кривой $R(t)$ в точке $Q_{k-1}$, значение $t_{k+1}$ следует выбирать
следующим образом:

$$
t_{k+1}=t_k+(t_k-t_{k-1}).
$$
