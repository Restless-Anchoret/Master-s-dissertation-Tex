\section{Основные теоремы}

В данном разделе даны упрощённые формулировки теорем о деформации и сглаживании кривых, приведённых в 7"~й главе
монографии А.П.~Побегайло~\cite{pobegaylo}. В этой же монографии можно найти подробные доказательства данных теорем.

\subsection*{Теорема о деформации кривых}

Рассмотрим произвольную мультипликативную группу $G$. Пусть $g_1(u)$ и $g_2(u)$ "--- некоторые
непрерывно-дифференцируемые до произвольного порядка кривые в $G$, причём $u \in [0,1]$ и выполняются следующие условия:

\begin{equation}
g_1(0)=e, \quad g_2(0)=e,
\label{deformed-curves-conditions}
\end{equation}

\noindent где $e$ "--- единичный элемент группы $G$.

Требуется построить такую непрерывно-дифференцируемую до произвольного порядка кривую $g(u)\in G$, где $u \in [0,1]$,
чтобы выполнялись следующие условия:

\begin{equation}
g^{(i)}(0)=g_1^{(i)}(0), \quad g^{(i)}(1)=g_2^{(i)}(1),
\label{deformation-curve-conditions}
\end{equation}

\noindent где $i \in \{0,1,2,\dots,k\},\quad k\in\mathbb{N}$.

\begin{definition}
Кривая $g(u)$ называется \textbf{деформацией} кривых $g_1(u)$ и $g_2(u)$. Другими словами, кривая $g(u)$ гладко
деформируется и переходит из кривой $g_1(u)$ в кривую $g_2(u)$.
\end{definition}

Теперь определим полиномы $\tau_k^-(u)$ и $\tau_k^+(u)$ для $u \in [0,1]$ по следующим формулам:

$$
\tau_k^-(u)=(1-\omega_k(u))u, \quad \tau_k^+(u)=\omega_k(u)u.
$$

\begin{theorem}
Если $g_1(u)$ и $g_2(u)$, где $u \in [0,1]$, являются такими непрерывно-дифференцируемыми до произвольного порядка
кривыми, которые удовлетворяют равенствам (\ref{deformed-curves-conditions}), то кривая

$$
g(u)=g_2(\tau_k^+(u))g_1(\tau_k^-(u)),
$$

\noindent где $u \in [0,1]$, удовлетворяет граничным условиям (\ref{deformation-curve-conditions}).
\end{theorem}

\subsection*{Следствие теоремы о деформации кривых}

Рассмотрим произвольную мультипликативную группу $G$. Пусть $g_1(u)$ и $g_2(u)$ "--- некоторые
непрерывно-дифференцируемые до произвольного порядка кривые в $G$, причём $u \in [0,1]$ и выполняются следующие условия:

\begin{equation}
g_1(0)=e, \quad g_2(0)=e,
\label{extruded-curves-conditions}
\end{equation}

\noindent где $e$ "--- единичный элемент группы $G$.

Обозначим также кривую $\tilde g_2(u)$ следующим образом:

$$
\tilde g_2(u)=g_2(1-u).
$$

Требуется построить такую непрерывно-дифференцируемую до произвольного порядка кривую $g(u)\in G$, где $u \in [0,1]$,
чтобы выполнялись следующие условия:

$$
g(0)=g_1(0), \quad g(1)=g_1(1),
$$

\begin{equation}
g^{(i)}(0)=g_1^{(i)}(0), \quad g^{(i)}(1)=\tilde g_2^{(i)}(1),
\label{extrusion-curve-conditions}
\end{equation}

\noindent где $i \in \{1,2,\dots,k\},\quad k\in\mathbb{N}$.

\begin{definition}
Кривая $g(u)$ называется результатом \textbf{вытягивания} кривой $g_1(u)$ посредством кривой $g_2(u)$.
\end{definition}

Теперь определим полиномы $\lambda_k^-(u)$ и $\lambda_k^+(u)$ для $u \in [0,1]$ по следующим формулам:

$$
\lambda_k^-(u)=(1-\omega_k(1-u))(1-u), \quad \lambda_k^+(u)=\omega_k(u)u.
$$

\begin{consequence}
Если $g_1(u)$ и $g_2(u)$, где $u \in [0,1]$, являются такими непрерывно-дифференцируемыми до произвольного порядка
кривыми, которые удовлетворяют равенствам (\ref{extruded-curves-conditions}), то кривая

$$
g(u)=g_2(\lambda_k^-(u))g_1(\lambda_k^+(u)),
$$

\noindent где $u \in [0,1]$, удовлетворяет граничным условиям (\ref{extrusion-curve-conditions}).
\end{consequence}

\subsection*{Теорема о сглаживании кривых}

Рассмотрим произвольную мультипликативную группу $G$. Пусть $g_1(u)$ и $g_2(u)$ "--- некоторые
непрерывно-дифференцируемые до произвольного порядка кривые в $G$, причём $u \in [0,1]$ и выполняются следующие условия:

\begin{equation}
g_1(0)=e, \quad g_2(0)=e,
\label{smoothed-curves-conditions}
\end{equation}

\noindent где $e$ "--- единичный элемент группы $G$.

Обозначим также кривую $\tilde g_1(u)$ следующим образом:

$$
\tilde g_1(u)=g_1(1-u).
$$

Требуется построить такую непрерывно-дифференцируемую до произвольного порядка кривую $g(u)\in G$, где $u \in [0,1]$,
чтобы выполнялись следующие условия:

\begin{equation}
g^{(i)}(0)=\tilde g_1^{(i)}(0), \quad g^{(i)}(1)=g_2^{(i)}(1),
\label{smoothing-curve-conditions}
\end{equation}

\noindent где $i \in \{0,1,2,\dots,k\},\quad k\in\mathbb{N}$.

\begin{definition}
Кривая $g(u)$ называется \textbf{сглаживанием} кривых $g_1(u)$ и $g_2(u)$. Другими словами, кривая $g(u)$ гладко
сглаживает точку соединения кривых $g_1(u)$ и $g_2(u)$.
\end{definition}

Теперь определим полиномы $\sigma_k^-(u)$ и $\sigma_k^+(u)$ для $u \in [0,1]$ по следующим формулам:

$$
\sigma_k^-(u)=(1-\omega_k(u))(1-u), \quad \sigma_k^+(u)=\omega_k(u)u.
$$

\begin{theorem}
Если $g_1(u)$ и $g_2(u)$, где $u \in [0,1]$, являются такими непрерывно-дифференцируемыми до произвольного порядка
кривыми, которые удовлетворяют равенствам (\ref{smoothed-curves-conditions}), то кривая

$$
g(u)=g_2(\sigma_k^+(u))g_1(\sigma_k^-(u)),
$$

\noindent где $u \in [0,1]$, удовлетворяет граничным условиям (\ref{smoothing-curve-conditions}).
\end{theorem}
