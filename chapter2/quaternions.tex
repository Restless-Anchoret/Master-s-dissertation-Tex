\section{Кватернионы}

Введём понятие кватерниона. Для этого рассмотрим следующие квадратные матрицы второго порядка:

$$
I=
\left[ { \begin{array}{cc}
    1 & 0 \\
    0 & 1 \\
\end{array} }\right],
\quad
E=
\left[ { \begin{array}{cc}
    0 & 1 \\
    -1 & 0 \\
\end{array} }\right].
$$

Помимо этого нам также необходимо определить следующие блочно-диаго\-нальные матрицы четвёртого порядка:

$$
1=
\left[ { \begin{array}{cc}
    I & 0 \\
    0 & I \\
\end{array} }\right],
\quad
i=
\left[ { \begin{array}{cc}
    0 & I \\
    -I & 0 \\
\end{array} }\right],
\quad
j=
\left[ { \begin{array}{cc}
    E & 0 \\
    0 & -E \\
\end{array} }\right],
\quad
k=
\left[ { \begin{array}{cc}
    0 & E \\
    E & 0 \\
\end{array} }\right].
$$

\begin{definition}
\textbf{Кватернионом} называется произвольная линейная комбинация вида:

$$
Q=q_0+q_1i+q_2j+q_3k
$$

\noindent матриц $1,i,j,k$. Числа $q_0,q_1,q_2,q_3$ называются \textbf{координатами кватерниона}. Действительное
число $q_0$ называется \textbf{скалярной частью кватерниона}, а линейная комбинация $q=q_1i+q_2j+q_3k$ "---
\textbf{векторной частью кватерниона}.
\end{definition}

Таким образом, кватернион $Q$ может быть представим в виде:

$$
Q=q_0+q.
$$

Множество всех кватернионов будем обозначать $H^4$. В этом множестве вводятся различные линейные операции. Также
необходимо дать несколько дополнительных определений.

\begin{definition}
\textbf{Сумма} двух кватернионов $P$ и $Q$ определяется по следующей формуле:

$$
P+Q=(p_0+q_0)+(p_1+q_1)i+(p_2+q_2)j+(p_3+q_3)k.
$$
\end{definition}

\begin{definition}
\textbf{Произведение} кватерниона $Q$ \textbf{на действительное число} $\lambda$ определяется по следующей формуле:

$$
\lambda Q=\lambda(q_0+q_1i+q_2j+q_3k)=\lambda q_0+\lambda q_1i+\lambda q_2j+\lambda q_3k.
$$
\end{definition}

\begin{definition}
\textbf{Произведение} двух кватернионов $P$ и $Q$ определяется по следующей формуле:

$$
RQ=(p_0+p)(q_0+q)=(p_0q_0-p \cdot q)+(p_0q+q_0p+p\times q).
$$
\end{definition}

\begin{definition}
Кватернионом, \textbf{сопряжённым} c кватернионом $Q$, называется кватернион:

$$
Q^*=(q_0+q)^*=q_0-q.
$$
\end{definition}

\begin{definition}
\textbf{Скалярным произведением} двух кватернинов $P$ и $Q$ называется действительное число:

$$
P \cdot Q=p_0q_0+p_1q_1+p_2q_2+p_3q_3.
$$
\end{definition}

\begin{definition}
\textbf{Нормой} кватерниона называется действительное число:

$$
N(Q)=Q \cdot Q.
$$
\end{definition}

\begin{definition}
\textbf{Векторное произведение} трёх кватернионов $P$, $Q$ и $R$ определяется по следующей формуле:

$$
P \times Q \times R=
\left| { \begin{array}{cccc}
    1 & i & j & k \\
    p_0 & p_1 & p_2 & p_3 \\
    q_0 & q_1 & q_2 & q_3 \\
    r_0 & r_1 & r_2 & r_3 \\
\end{array} }\right|
$$
\end{definition}

Таким образом, можно показать, что множество кватернионов $H^4$ образует \textit{алгебру кватернионов} с введёнными
выше операциями сложения, умножения и умножения на действительное число.

\begin{definition}
Кватернион $Q$ называется \textbf{единичным}, если он удовлетворяет условию:

$$
N(Q)=1.
$$
\end{definition}

Можно показать, что множество всех единичных кватернионов образует мультипликативную группу вместе с операцией
умножения и кватернионом $1$ в качестве единицы.

Единичные кватернионы так же, как и собственно ортогональные матрицы, удобно использовать для обозначения поворотов и
ориентаций объектов в трёхмерном пространстве, хотя мы будем в основном обозначать ими ориентации.

Пусть $n \in \mathbb{R}^3$ "--- единичный вектор, и $\varphi \in \mathbb{R}$ "--- некоторый угол. Тогда ориентация
объекта, в которую он может перейти из базовой ориентации в результате поворота относительно оси~$n$ на угол~$\varphi$,
определяется следующей формулой:

$$
Q(n,\varphi)=\cos\frac{\varphi}{2}+\sin\frac{\varphi}{2}n.
$$

Пусть теперь $p$ "--- точка трёхмерного объекта, определённая в системе координат, заданной относительно точки опоры
объекта. Определим кватернион $P$, как кватернион, имеющий нулевую скалярную часть и векторную часть, равную $p$.
Тогда определить координаты точки $p$ после поворота вокруг оси $n$ на угол $\varphi$ можно по формуле:

$$
\tilde P=QPQ^*.
$$

В данной формуле $\tilde P$ также имеет нулевую скалярную часть, а его векторная часть $\tilde p$ будет иметь искомые
координаты точки $p$ после поворота.
