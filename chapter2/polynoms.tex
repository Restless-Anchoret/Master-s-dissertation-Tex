\section{Сглаживающие полиномы}

\begin{definition}
\textbf{Полиномами Бернштейна} степени $n$ называются полиномы $b_{n,m}(u)$ следующего вида:

$$
b_{n,m}(u)=C_n^m(1-u)^{n-m}u^m.
$$
\end{definition}

Здесь $n \ge 0, m \ge 0, u \in [0,1]$, а $C_n^m$ обозначает количество сочетаний:

$$
C_n^m=\frac{n!}{m!(n-m)!}
$$

Наиболее часто используются полиномы Бернштейна младших степеней:

$$
b_{1,0}(u)=1-u, \quad b_{1,1}(u)=u,
$$

$$
b_{2,0}(u)=(1-u)^2, \quad b_{2,1}(u)=2(1-u)u, \quad b_{2,2}(u)=u^2,
$$

$$
b_{3,0}(u)=(1-u)^3, \quad b_{3,1}(u)=3(1-u)^2u, \quad b_{3,2}(u)=3(1-u)u^2, \quad b_{3,3}(u)=u^3.
$$

\begin{definition}
\textbf{Сглаживающими полиномами} степени $k$ называются полиномы $\omega_k(u)$ следующего вида:

$$
\omega_k(u)=\sum_{i=k+1}^{2k+1}b_{2k+1,i}(u)
$$
\end{definition}

Здесь $k \ge 0, u \in [0,1]$.

Характерная особенность таких полиномов заключается в том, что они сглаживают ступенчатую функцию. Наиболее часто
используются сглаживающие полиномы младших степеней:

$$
\omega_0(u)=u,
$$

$$
\omega_1(u)=3(1-u)u^2+u^3,
$$

$$
\omega_2(u)=10(1-u)^2u^3+5(1-u)u^4+u^5.
$$
