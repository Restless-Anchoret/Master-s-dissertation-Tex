В данной главе будет подробно описана теория, которая будет использована в последующих главах при описании алгоритмов
построения сплайн-кривых.

В первом разделе данной главы речь пойдёт о представлении поворотов трёхмерных объектов, собственно ортогональных
матрицах и двумерной сфере~$S^2$. Во втором разделе "--- о представлении ориентаций трёхмерных объектов, кватернионах
и ориентационной сфере~$S^3$. В третьем разделе будут даны определения полиномов Бернштейна и сглаживающих полиномов,
а в четвёртом будут сформулированы основные теоремы, на которые мы будем опираться в дальнейшем: теорема о деформации
кривых и теорема о сглаживании кривых.
