В данной главе будет подробно описана теория, которая будет использована в последующих главах при описании алгоритмов
построения сплайн-кривых.

В первом и втором разделах данной главы речь пойдёт о представлении поворотов трёхмерных объектов, собственно
ортогональных матрицах и двумерной сфере~$S^2$. В третьем и четвёртом разделах "--- о представлении ориентаций
трёхмерных объектов, кватернионах и ориентационной сфере~$S^3$. В пятом разделе будут даны определения полиномов
Бернштейна и сглаживающих полиномов, а в шестом будут сформулированы основные теоремы, на которые мы будем
опираться в дальнейшем: теорема о деформации кривых и теорема о сглаживании кривых.
