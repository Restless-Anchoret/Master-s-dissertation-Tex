\section{Двумерная сфера}

Пусть $p_0=(x_0,y_0,z_0)$ "--- произвольная точка из $\mathbb{R}^3$, и $r$ "--- произвольное положительное
действительное число. Определим множество точек $(x,y,z)$, удовлетворяющих уравнению:

\begin{equation}
(x-x_0)^2+(y-y_0)^2+(z-z_0)^2=r^2.
\label{two-dimenstion-sphere}
\end{equation}

\begin{definition}
Множество точек, удовлетворяющих уравнению~(\ref{two-dimenstion-sphere}), называется \textbf{двумерной сферой}
радиуса~$r$ с центром в точке $p_0$. Двумерную сферу принято обозначать $S^2$.
\end{definition}

В дальнейшем будем предполагать, что центр сферы совпадает с центром используемой системы координат, т.~е. $p_0=(0,0,
0)$. Тогда уравнение двумерной сферы примет вид:

\begin{equation}
x^2+y^2+z^2=r^2.
\label{canonical-two-dimension-sphere}
\end{equation}

\begin{definition}
Уравнение~(\ref{canonical-two-dimension-sphere}) называется \textbf{каноническим уравнением двумерной сферы}.
\end{definition}

Пусть теперь $S^2$ "--- двумерная сфера, описанная каноническим уравнением относительно некоторой ортогональной
системы координат. Выберем произвольную плоскость, заданную относительно той же системы координат следующим уравнением:

\begin{equation}
n \cdot p = d,
\label{plane}
\end{equation}

\noindent где $p=(x,y,z)$ "--- произвольная точка плоскости, $|n|=1$, и $0 \le d < r$.

Тогда $n$ будет задавать перпендикуляр к данной плоскости, а $d$ будет равно расстоянию от этой плоскости до центра
системы координат.

\begin{definition}
Если $d=0$, то пересечение сферы $S^2$ и плоскости, заданной уравнением~(\ref{plane}), называется \textbf{большой
окружностью} сферы $S^2$, а непрерывное подмножество этого пересечения "--- \textbf{большой дугой}.
\end{definition}

\begin{definition}
Если $0<d<r$, то пересечение сферы $S^2$ и плоскости, заданной уравнением~(\ref{plane}), называется \textbf{малой
окружностью} сферы $S^2$, а непрерывное подмножество этого пересечения "--- \textbf{малой дугой}.
\end{definition}

Предположим теперь, что $p_0$ "--- точка, лежащая на большой или малой дуге окружности $S^2$, и $\varphi$ "--- угол,
задающий длину дуги ($-\pi~\le~\varphi~\le~\pi$). Тогда дугу можно задать следующей формулой:

$$
p(u)=R(n,u\varphi)p_0, \quad u \in [0,1].
$$
