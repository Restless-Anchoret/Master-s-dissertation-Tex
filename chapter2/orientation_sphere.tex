\section{Ориентационная сфера}

Пусть $P_0=(w_0,x_0,y_0,z_0)$ "--- произвольная точка из $\mathbb{R}^4$. Определим множество точек $(w,x,y,z)$,
удовлетворяющих уравнению:

\begin{equation}
(w-w_0)^2+(x-x_0)^2+(y-y_0)^2+(z-z_0)^2=1.
\label{orientation-sphere}
\end{equation}

\begin{definition}
Множество точек, удовлетворяющих уравнению~(\ref{orientation-sphere}), называется \textbf{трёхмерной единичной} или
\textbf{ориентационной сферой} с центром в точке $P_0$. Ориентационную сферу принято обозначать $S^3$.
\end{definition}

В дальнейшем будем предполагать, что центр сферы совпадает с центром используемой системы координат, т.~е. $P_0=(0,0,
0,0)$. Тогда уравнение ориентационной сферы примет вид:

\begin{equation}
w^2+x^2+y^2+z^2=1.
\label{canonical-orientation-sphere}
\end{equation}

\begin{definition}
Уравнение~(\ref{canonical-orientation-sphere}) называется \textbf{каноническим уравнением ориентационной сферы}.
\end{definition}

Точки ориентационной сферы можно равноправно рассматривать как векторы из пространства $\mathbb{R}^4$ и как единичные
кватернионы. Вообще говоря, так как ориентационная сфера полностью состоит из единичных кватернионов, то она, по
сути, представляет собой множество всевозможных ориентаций объекта в трёхмерном пространстве.

В данном разделе для введения определений мы будем рассматривать точки ориентационной сферы как векторы из
$\mathbb{R}^4$, а впоследствии в уравнениях "--- как единичные кватернионы.

Пусть теперь $S^3$ "--- ориентационная сфера, описанная каноническим уравнением относительно некоторой ортогональной
системы координат. Выберем произвольную плоскость, заданную относительно той же системы координат как пересечение
двух гиперплоскостей:

\begin{equation}
m \cdot p=d_1, \quad l \cdot p=d_2,
\label{hiberplane}
\end{equation}

\noindent где $d_1,d_2,m,l$ удовлетворяют следующим соотношениям:

$$
d_1 \ge 0, \quad d_2 \ge 0, \quad 0 \le d_1^2+d_2^2<1,
$$

$$
|m|=1, \quad |l|=1, \quad m \cdot l = 0.
$$

\begin{definition}
Если $d_1=0$ и $d_2=0$, то пересечение сферы $S^3$ и плоскости, заданной уравнением~(\ref{hiberplane}), называется
\textbf{большой окружностью} сферы $S^3$, а непрерывное подмножество этого пересечения "--- \textbf{большой дугой}.
\end{definition}

\begin{definition}
Если $0<d_1^2+d_2^2<1$, то пересечение сферы $S^3$ и плоскости, заданной уравнением~(\ref{hiberplane}), называется
\textbf{малой окружностью} сферы $S^3$, а непрерывное подмножество этого пересечения "--- \textbf{малой дугой}.
\end{definition}

Предположим теперь, что $P_0$ "--- единичный кватернион, лежащий на большой или малой дуге окружности $S^3$, $\phi$
"--- угол, задающий длину дуги ($-\pi~\le~\phi~\le~\pi$), а $n$ "--- ось. Тогда дугу можно задать следующей
формулой:

$$
P(u)=Q(n,u\phi)P_0=(\cos\frac{\phi}{2}+\sin\frac{\phi}{2}n)P_0, \quad u \in [0,1].
$$

В результате дуга $P(u)$ будет содержать в себе все ориентации, через которые пройдёт трёхмерный объект при повороте
из ориентации $P_0$ на угол $\phi$ вокруг оси $n$.
