\section{Модуль рендеринга}

Модуль рендеринга содержит в себе всего три пакета:

\begin{itemize}

\item \texttt{world}. Содержит набор вложенных друг в друга классов, представляющих собой подробное описание так
называемых миров с 3D-объектами для рендеринга.
\item \texttt{control}. Содержит интерфейс \texttt{Control} и несколько его реализаций. Этот интерфейс представляет
собой своего рода внешний рычаг управления некоторыми параметрами внутри миров.
\item \texttt{core}. Содержит основную логику рендеринга, т.~е. реализацию математических алгоритмов отрисовки
3D-объектов на экране.

\end{itemize}

Рассмотрим содержимое этих пакетов чуть более подробно, обсудив, каким образом следует использовать данный модуль
в коде-клиенте.

При использовании данного модуля нужно подготовить свою реализацию интерфейса \texttt{World\-Factory}, который имеет
всего один метод \texttt{World create\-World()}. Чтобы его реализовать, т.~е. чтобы создать объект класса \texttt{World},
нужно подготовить изначальное положение камеры (объект класса \texttt{Camera}) и набор алгоритмов для создания
3D-объектов (реализаций интерфейса \texttt{World\-ObjectCreator}).

Реализация интерфейса \texttt{WorldObjectCreator}, в свою очередь, предполагает создание связанных между собой набора
реализаций интерфейса \texttt{Con\-trol} и объекта класса \texttt{WorldObjectContent}. Последний содержит описание
анимации объекта в виде класса \texttt{AnimationInfo} (содержащего функцию, выражающую зависимость ориентации объекта
от времени), и описание отдельных частей объекта. Каждая часть содержит набор координат вершин объекта и набор пар
индексов вершин, соединённых рёбрами, а также "--- цвет, толщину ребра в пикселях и радиус вершины в пикселях.

Таким образом, реализация интерфейса \texttt{World\-Factory} должна содержать в себе полное описание мира, включая
алгоритмы создания каждого объекта и рычаги управления параметрами в мире.

Когда создано несколько миров, т.~е. наследников класса \texttt{World}, их можно передать в класс
\texttt{RenderingEngine}. Также этот класс принимает реализацию интерфейса \texttt{RenderingDelegate}, служащего точкой
расширяемости данного модуля. Это означает, что реализация может быть произвольной: как использующей библиотеку
OpenGL, как в нашем случае, так и использующей любую другую библиотеку. Реализация должна задавать только отрисовку
самых элементарных графических элементов, а именно "--- отрезка и круга.

Далее в \texttt{RenderingEngine} можно добавить произвольное количество реализаций интерфейса \texttt{RenderingAction}.
В каждой реализации можно отрисовывать что-либо на экране, при этом каждая следующая реализация будет отрисовывать
поверх того, что было отрисовано ранее.

По умолчанию используется только одна реализация "--- \texttt{WorldRende\-ring\-Ac\-tion}. Эта реализация вычисляет
координаты 3D-объектов одного из миров, спроецированных на экран перед камерой по правилам перспективы,
и производит соответсвующую отрисовку. Поэтому удобно добавлять дополнительные реализации \texttt{RenderingAc\-tion}
поверх той, что идёт по умолчанию, чтобы выводить на экран различную метаинформацию.
