\section{Модуль алгебраических объектов}

Данный модуль состоит из следующих пакетов:

\begin{itemize}

\item \texttt{common}. Содержит интерфейс \texttt{Algebraic\-Object}, стоящий в самом верху иерархии классов,
реализующих различные алгебраические объекты. Также в данном пакете лежит несколько вспомогательных классов
с реализацией решений некоторых алгебраических задач.
\item \texttt{exception}. Содержит некоторые классы-исключения для обозначениях различных алгебраических ошибок.
\item \texttt{function}. Содержит класс \texttt{Double\-Function<T>} и наследующий от него класс
\texttt{Double\-Multifunction<T>}. Оба класса реализуют интерфейс \texttt{Al\-gebraic\-Object}, т.~е. функции можно
складывать, перемножать, делать над ними некоторые другие операции и получать в результате новые функции.
\item \texttt{line}. Содержит единственный класс \texttt{Line} для задания прямой на плоскости.
\item \texttt{matrix}. Содержит единственный класс \texttt{Matrix}, реализующий интерфейс \texttt{Al\-gebraic\-Object} и
используемый для совершения операций над матрицами.
\item \texttt{quaternion}. Содержит единственный класс \texttt{Quaternion}, реализующий интерфейс
\texttt{Algebraic\-Object} и используемый для совершения операций над кватернионами.
\item \texttt{vector}. Содержит классы \texttt{Double\-Vector}, \texttt{Two\-Double\-Vector},
\texttt{Three\-Double\-Vec\-tor}, \texttt{Int\-Vector}, \texttt{Two\-Int\-Vector}, \texttt{Three\-Int\-Vector} и
\texttt{Sin\-gle\-Double}. Все они реализуют интерфейс \texttt{Algebraic\-Object} и используются для совершения операций
над векторами.

\end{itemize}

Отдельно здесь стоит сделать акцент на значимости интерфейса \texttt{Algebra\-ic\-Object}. Ниже приведён список методов,
которые он содержит:

\begin{itemize}

\item \texttt{T add(T other)} "--- сложение с другим алгебраическим объектом того же типа;
\item \texttt{T substract(T other)} "--- вычитание другого алгебраического объекта того же типа;
\item \texttt{T multiply(int number)} "--- умножение на целое число;
\item \texttt{T multiply(double number)} "--- умножение на вещественное число;
\item \texttt{T multiply(T other)} "--- умножение на другой алгебраический объект того же типа;
\item \texttt{T elementWiseMultiply(T other)} "--- поэлементное умножение на другой алгебраический объект того же типа;
\item \texttt{double scalarMultiply(T other)} "--- скалярное умножение на другой алгебраический объект того же типа;
\item \texttt{double getNorm()} "--- вычисление нормы данного алгебраического объекта;
\item \texttt{T normalized()} "--- нормализация данного алгебраического объекта;
\item \texttt{boolean isZero()} "--- возвращает, является ли данный алгебраический объект нулевым.

\end{itemize}

Поскольку все классы алгебраических объектов реализуют этот интерфейс, это позволяет единообразно обращаться с ними.
Кроме того, особенное удобство возникает при использовании generic-класса \texttt{Double\-Function<T>}, который может
являться функцией от вещественной переменной, возвращающей алгебраический объект произвольного типа, будь то
вещественное число, вектор, матрица или кватернион. Благодаря тому, что сам класс \texttt{Double\-Function<T>}
реализует интерфейс алгебраического объекта, реализация сложных формул с вложенными друг в друга функциями, становится
намного проще.
