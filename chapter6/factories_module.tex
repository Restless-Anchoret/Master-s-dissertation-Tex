\section{Модуль алгоритмов}

Данный модуль представляет наибольший интерес в рамках данной работы, так как именно он содержит реализацию всех
алгоритмов построения сплайн-кривых, рассмотренных в предыдущих главах.

Модуль состоит из следующих пакетов:

\begin{itemize}

\item \texttt{interpolation}. Содержит реализованные алгоритмы построения сплайн-кри\-вых, а также используемые ими
вспомогательные алгоритмы.
    \begin{itemize}
    \item \texttt{tools}. Содержит ряд вспомогательных алгоритмов, таких как: построение отрезка, окружности и
    касательной на плоскости, построение малых, больших и касательных дуг на двумерной и ориентационной сферах.
    \item \texttt{curvecreators}. Содержит непосредственно алгоритмы построения сплайн-кривых.
    \item \texttt{input}. Содержит классы с рядом входных параметров для каждого из алгоритмов.
    \item \texttt{exception}. Содержит классы-исключения для обозначения ошибок при интерполировании кривых.
    \end{itemize}

\item \texttt{figures}. Содержит ряд классов-фабрик для создания фигур 3D-объектов. Под фигурой в данном случае
подразумевается только набор координат вершин и набор пар индексов вершин, соединённых рёбрами.

\item \texttt{animations}. Содержит ряд классов-фабрик для создания анимаций 3D-объек\-тов. Под анимацией в данном
случае подразумевается функция, отображающая вещественные числа (время) в группу единичных кватернионов (ориентации).

\item \texttt{world}. Содержит ряд классов-фабрик для создания 3D-миров. В этих классах создаются примеры-демонстрации
результатов каждого из алгоритмов, рассмотренных в предыдущих главах.

\item \texttt{util}. Содержит некоторые вспомогательные классы для таких операций, как преобразование системы
координат или создания абстрактной операции умножения в мультипликативной группе.

\item \texttt{constants}. Содержит классы для создания наборов констант, использующихся при создании 3D-миров.

\end{itemize}

Некоторые классы из пакета \texttt{interpolation}, в том числе реализующие алгоритмы построения сплайн-кривых,
приведены в приложениях к данной работе.
