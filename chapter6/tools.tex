\section{Средства реализации}

В качестве языка программирования для реализации приложения выбран язык Java. Такой выбор объясняется удобством
использования объектно-ориентирован\-ного и функционального программирования в данном языке, а также "--- удобством
реализации паттернов проектирования~\cite{gamma}, которые будет уместно применить для расширяемости приложения.

При разработке приложения были использованы различные технологии. Ниже приведён полный список этих технологий:

\begin{itemize}
\item JDK 1.8~\cite{java} "--- выбрана версия 1.8, поскольку в ней поддерживаются лямбда-выражения, которые было весьма
удобно использовать при реализации;
\item LWJGL 2.8.2~\cite{lwjgl} "--- библиотека использована для интеграции Java с OpenGL и для реализации
графического движка;
\item Maven "--- использован в качестве фреймворка для сборки проекта;
\item Git "--- использован в качестве системы контроля версий;
\item GitHub "--- использован для хранения репозитория;
\item IntelliJ IDEA 2018.1.2 "--- данная IDE предоставляет гибкие средства для редактирования и рефакторинга кода.
\end{itemize}
