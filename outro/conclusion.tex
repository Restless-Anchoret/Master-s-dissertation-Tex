\chapter*{Заключение}
\addcontentsline{toc}{chapter}{Заключение}

В данной работе была поставлена задача исследовать известные и предложить новые алгоритмы построения сплайн-кривых
на плоскости, двумерной и ориентационной сферах, а также реализовать все рассмотренные алгоритмы в виде приложения
с 3D-визуализацией.

В ходе работы были подробно рассмотрены методы построения сплайн-кри\-вых, предложенные известным математиком
А.П.~Побегайло \cite{pobegaylo} и основанные на использовании свойств полиномов Бернштейна и сглаживающих полиномов,
а также на использовании теорем о деформации и сглаживании кривых. Эти методы были применены к плоскости, и, более того,
на их основе были разработаны новые методы построения сплайн-кривых с дополнительными условиями, налагаемыми на
сплайн-кривые, а именно "--- направлениями касательных в точках.

Все рассмотренные алгоритмы были реализованы в виде приложения с 3D-визуализацией. Приложение было написано на языке
Java с использованием библиотеки LWJGL, оболочки над низкоуровневой библиотекой OpenGL. Для каждой рассмотренной задачи
были смоделированы объекты, демонстрирующие результаты интерполирования сплайн-кривых по описанным алгоритмам.
В результате проделанной работы мы смогли убедиться, что кривые, построенные по рассмотренным методам, удовлетворяют
заявленным требованиям, т.~е. являются непрерывно-дифференцируемыми до заданного порядка.

Как говорилось ранее, тематика, затронутая в данной работе, может считаться актуальной сегодня, а полученные в ходе
данной работы результаты могут найти применение в областях компьютерной графики и робототехники.
