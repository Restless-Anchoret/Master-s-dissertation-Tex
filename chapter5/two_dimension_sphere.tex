\section{На двумерной сфере}

Рассмотрим задачу построения кривой по набору точек и направлениям касательных на поверхности двумерной сферы.
Дан набор $k$ точек $\{p_i\}$ на поверхности $S^2$ и $k$ углов $\{\phi_i\}$, причём некоторые $\phi_i$ могут быть
не заданы. Нужно построить непрерывно-дифференцируемую до заданного порядка кривую, лежащую на поверхности сферы $S^2$ и
проходящую последовательно через все эти точки под заданными углами к экватору $S^2$.

\subsection*{Построение дуги-касательной}

Поговорим вначале о том, как построить дугу на поверхности сферы $S^2$ в направлении касательной, заданной под углом к
экватору сферы.

Пусть дана точка $p=(x,y,z)$, лежащая на поверхности сферы $S^2$, и задан некоторый угол $\phi$. Пусть этот угол к
экватору $S^2$ задаёт направление касательной. Определим две кривые $p(u)$ и $q(u)$, лежащие на поверхности $S^2$,
как результаты поворота точки $p$, соответственно, в направлении касательной на угол $\psi$, и в направлении,
обратном направлению касательной, на угол $\chi$.

Для этого вначале определим проекцию $p_0$ точки $p$ на плоскость $xOy$:

$$
p_0=(x,y,0).
$$

Далее определим вектор $a_0$, который будет лежать в плоскости, параллельной $xOy$, и будет касаться сферы $S^2$:

$$
a=R(z_0,-\frac{\pi}{2})p_0, \quad \text{где} \quad z_0=(0,0,1),
$$

$$
a_0=\frac{a}{||a||}.
$$

Определим также вектор $b_0$, касающийся сферы $S^2$, и выходящий из точки $p$ в направлении самой верхней точки сферы:

$$
b=p_0 \times a_0, \quad b_0=\frac{b}{||b||}.
$$

Теперь мы можем определить ось $n$, вокруг которой будут осуществляться повороты точки $p$:

$$
n=a_0\sin\phi+b_0(-\cos\phi).
$$

Наконец, определим кривые $p(u)$ и $q(u)$:

$$
p(u)=R(n,u\psi)p_0, \quad q(u)=R(n,-u\chi)p_0.
$$

\subsection*{Построение кривой с заданными направлениями \mbox{касательных} в точках}

Пусть теперь есть две различные точки $p_1$ и $p_2$ на поверхности сферы $S^2$, соединённые некоторой
непрерывно-дифферецируемой до любого порядка дугой $p(u)$. Пусть кривая $p(u)$ будет результатом поворота точки $p_1$
вокруг оси $n$ на угол $\alpha$. Также для точек $p_1$ и $p_2$ задана хотя бы одна из дуг-касательных $q_1(u)$
и $q_2(u)$, как было описано в предыдущем подразделе. Дуги-касательные будут результатами поворотов точек $p_1$ и $p_2$
вокруг осей $m_1$ и $m_2$ на углы $\psi_1$ и $\psi_2$.

Таким образом:

$$
p(u)=R(n,u\alpha)p_1, \quad q_1(u)=R(m_1,u\psi_1)p_1, \quad R(m_2,u\psi_2)p_2.
$$

Если задана дуга $q_2(u)$, введём дополнительное обозначение:

$$
q_3(u)=q_2(1-u), \quad u \in [0,1].
$$

Требуется провести на поверхности сферы $S^2$ кривую $r(u)$, соединяющую точки $p_1$ и $p_2$ и удовлетворяющую
следующим требованиям:

\begin{equation*}
r^{(k)}(0)=
  \begin{cases}
    q_1^{(k)}(0), & \text{если $q_1(u)$ задана}, \\
    p^{(k)}(0),   & \text{иначе};
  \end{cases}
\end{equation*}

\begin{equation*}
r^{(k)}(1)=
  \begin{cases}
    q_3^{(k)}(1), & \text{если $q_2(u)$ задана}, \\
    p^{(k)}(1),   & \text{иначе};
  \end{cases}
\end{equation*}

$$
k \in {0,1,\dots,l}, \quad l \in \mathbb{N}.
$$

Рассмотрим три возможных случая.

\bigskip
\textit{Случай 1.}

Пусть угол $\phi_1$ задан, а угол $\phi_2$ "--- нет.

Тогда следует только построить деформацию из кривой $q_1(u)$ в кривую $p(u)$. Итоговая кривая $r(u)$ может быть
выражена следующей формулой:

$$
r(u)=R(n,w_l(u)u\alpha)R(m_1,(1-w_l(u))u\psi_1)p_1, \quad u \in [0,1].
$$

Отсюда выразим функцию $R_1(u)$, описывающую для каждого $u$ некоторый поворот точки $p_1$:

\begin{equation}
R_1(u)=R(n,w_l(u)u\alpha)R(m_1,(1-w_l(u))u\psi_1), \quad u \in [0,1].
\label{two-dimension-first-deformed-curve}
\end{equation}

\bigskip
\textit{Случай 2.}

Пусть угол $\phi_2$ задан, а угол $\phi_1$ "--- нет.

Тогда следует построить вытягивание поворота $R(n,u\alpha)$ посредством поворота $R(m_2,u\psi_2)$. Итоговая кривая
$r(u)$ будет выражена следующей формулой:

$$
r(u)=R(m_2,(1-w_l(1-u))(1-u)\psi_2)R(n,w_l(u)u\alpha)p_1, \quad u \in [0,1].
$$

Отсюда выразим функцию $R_2(u)$, описывающую для каждого $u$ некоторый поворот точки $p_1$:

\begin{equation}
R_2(u)=R(m_2,(1-w_l(1-u))(1-u)\psi_2)R(n,w_l(u)u\alpha), \quad u \in [0,1].
\label{two-dimension-second-deformed-curve}
\end{equation}

\bigskip
\textit{Случай 3.}

Пусть теперь заданы оба угла $\phi_1$ и $\phi_2$.

Вначале вычислим функции $R_1(u)$ и $R_2(u)$ по формулам (\ref{two-dimension-first-deformed-curve}) и
(\ref{two-dimension-second-deformed-curve}). Далее необходимо лишь построить деформацию $R_1(u)$ в $R_2(u)$:

$$
R(u)=R_2(w_l(u)u)R_1((1-w_l(u))u).
$$

Итовая кривая $r(u)$ имеет вид:

$$
r(u)=R(u)p_1.
$$

\subsection*{Алгоритм построения кривой}

Приведём теперь детальное описание алгоритма построения кривой $r(t)$, проходящей последовательно через точки
$p_1,\dots,p_k$ под углами $\phi_1,\dots,\phi_k$.

\bigskip
\textit{Шаг} 0.

\begin{enumerate}
\item Построим малые дуги через каждые три последовательные точки $p_i$, $p_{i+1}$, $p_{i+2}$,
$i \in \{1,\dots,k-2\}$, вычислим кривые $s_i(u)$ и $v_i(u)$, $i \in \{1,\dots,k-2\}$ по формуле
(\ref{two-dimension-arc}), запомним углы соответствующих дуг, как $\psi_i$ и $\chi_i$.
\item Построим большие дуги, соединяющие каждые две последовательные точки $p_i$, $p_{i+1}$, $i \in \{1,\dots,k-1\}$,
вычислим кривые $z_i(u)$ по формуле (\ref{two-dimension-big-arc}), запомним при этом угол поворота дуги, как $\alpha_i$.
\item Построим дуги-касательные для тех точек $p_i$, $i \in \{1,\dots,k\}$, для которых задан угол $\phi_i$.
Вычислим дугу $q_i^+(u)$ для угла $\phi_i$ (прямое направление касательной), как результат поворота точки $p_i$
на угол $\alpha_i$, и дугу $q_i^-(u)$ для угла $\phi_i+\pi$ (обратное направление касательной), как результат поворота
точки $p_i$ на угол $-\alpha_{i-1}$.
\end{enumerate}

\bigskip
\textit{Шаг} 1.

Определим первый сегмент $r_1(t)$ кривой $r(t)$.

Если не заданы углы $\phi_1$ и $\phi_2$, достаточно определить $r_1(t)$ по формуле:

$$
r_1(t)=s_1(u(t)).
$$

Если задан хотя бы один из углов, следует использовать формулы из предыдущего подраздела. Применить их следует к
точкам $p_1$ и $p_2$, дугам-касательным $q_1^+(u)$ и $q_2^-(u)$, а в качестве кривой, соединяющей точки, нужно
выбрать либо $s_1(u)$, если не задан $\phi_2$, либо $z_1(u)$ в ином случае, вместе с соответствующими углами
поворотов и осями. После применения формулы в полученной записи нужно заменить $u$ на $u(t)$.

В итоговой записи $u(t)$ имеет следующее значение:

$$
u(t)=\frac{t-t_1}{t_2-t_1}, \quad t \in [t_1,t_2].
$$

Значения $t_1$ и $t_2$ следует выбирать, исходя из каких-либо дополнительных кинематических требований или ограничений.

\bigskip
\textit{Шаг} $i$, где $i \in \{2,3,\dots,k-2\}$.

Вычислим сегмент $r_i(t)$ кривой $r(t)$.

Если не заданы углы $\phi_i$ и $\phi_{i+1}$, достаточно определить $r_i(t)$, как деформацию $v_{i-1}(u)$ в $s_i(u)$.
Если задан хотя бы один из углов, следует использовать формулы из предыдущего подраздела. Применить их следует к
точкам $p_i$ и $p_{i+1}$, дугам-касательным $q_i^+(u)$ и $q_{i+1}^-(u)$, а в качестве кривой, соединяющей точки,
нужно выбрать либо $s_i(u)$, если не задан $\phi_{i+1}$, либо $v_{i-1}(u)$, если не задан $\phi_i$, либо $z_i(u)$
в ином случае, вместе с соответствующими углами поворотов и осями. После применения формулы в полученной записи нужно
заменить $u$ на $u(t)$.

В итоговой записи $u(t)$ имеет следующее значение:

$$
u(t)=\frac{t-t_i}{t_{i+1}-t_i}, \quad t \in [t_i,t_{i+1}].
$$

Чтобы обеспечить параметрическую непрерывность кривой $r(t)$ в точках $p_i$, значения $t_i$ следует выбирать
следующим образом:

$$
t_{i+1}=t_i+\frac{t_i-t_{i-1}}{\psi_{i-1}}\chi_{i-1}, \quad i \in \{2,3,\dots,k-2\}.
$$

\bigskip
\textit{Шаг} $(k-1)$.

Вычислим сегмент $r_{k-1}(t)$ кривой $r(t)$.

Если не заданы углы $\phi_{k-1}$ и $\phi_k$, достаточно определить $r_{k-1}(t)$ по формуле:

$$
r_{k-1}(t)=v_{k-2}(u(t)).
$$

Если задан хотя бы один из углов, следует использовать формулы из предыдущего подраздела. Применить их следует к
точкам $p_{k-1}$ и $p_k$, дугам-касательным $q_{k-1}^+(u)$ и $q_k^-(u)$, а в качестве кривой, соединяющей точки,
нужно выбрать либо $v_{k-2}(u)$, если не задан $\phi_{k-1}$, либо $z_{k-1}(u)$ в ином случае, вместе с соответствующими
углами поворотов и осями. После применения формулы в полученной записи нужно заменить $u$ на $u(t)$.

В итоговой записи $u(t)$ имеет следующее значение:

$$
u(t)=\frac{t-t_k}{t_{k+1}-t_k}, \quad t \in [t_k,t_{k+1}].
$$

Чтобы обеспечить параметрическую непрерывность кривой $r(t)$ в точке $p_{k-1}$, значение $t_{k+1}$ следует выбирать
следующим образом:

$$
t_k=t_{k-1}+\frac{t_{k-1}-t_{k-2}}{\psi_{k-2}}\chi_{k-2}.
$$
