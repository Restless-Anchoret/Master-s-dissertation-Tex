\section{На ориентационной сфере}

Рассмотрим задачу построения кривой по набору ориентаций и направлениям касательных на поверхности сферы единичных
кватернионов. Дан набор $k$ ориентаций $\{Q_i\}$ на поверхности $S^3$ и $k$ углов $\{\phi_i\}$, причём некоторые
$\phi_i$ могут быть не заданы. Нужно построить непрерывно-дифференцируемую до заданного порядка кривую, лежащую
на поверхности сферы $S^3$ и проходящую последовательно через все ориентации под заданными углами к экватору $S^3$.

\subsection*{Построение дуги-касательной на $S^3$}

Поговорим вначале о том, как построить дугу на поверхности сферы $S^3$ в направлении касательной, заданной под углом к
экватору сферы.

Пусть дана ориентация $Q$ и задан некоторый угол $\phi$ к экватору $S^3$, задающий направление касательной.
Определим две кривые $P(u)$ и $Q(u)$, лежащие на поверхности $S^3$, как результаты поворота из ориентации $Q$,
соответственно, в направлении касательной на угол $\psi$, и в направлении, обратном направлению касательной,
на угол $\chi$.

Вначале определим координаты точки $p=(x,y,z)$, в которую перейдёт точка $z_0=(0,0,1)$ после поворота из базовой
ориентации в ориентацию $Q$. Это можно сделать по следующей формуле:

$$
p=Qz_0Q^*,
$$

\noindent где $z_0$ используется, как кватернион с нулевой скалярной частью.

Далее определим ось $n$, вокруг которой должны осуществляться повороты в направлении, заданном углом $\phi$. Для этого
используем формулы (\ref{two-dimension-tangent-axis-1}), (\ref{two-dimension-tangent-axis-2}) и
(\ref{two-dimension-tangent-axis-3}).

После этого мы можем определить кривые $P(u)$ и $Q(u)$ следующим образом:

$$
P(u)=R(n,u\psi)Q, \quad Q(u)=R(n,-u\chi)Q.
$$

В этой записи, как и в остальных формулах до конца данного раздела, множители вида $R(n,\alpha)$ обозначают кватернионы,
задающие поворот вокруг оси $n$ на некоторый угол $\alpha$.

\subsection*{Построение кривой с заданными направлениями \mbox{касательных} в точках}

Пусть теперь есть две различные ориентации $Q_1$ и $Q_2$ на поверхности сферы $S^3$, соединённые некоторой
непрерывно-дифферецируемой до любого порядка дугой $P(u)$. Пусть кривая $P(u)$ будет результатом поворота из
ориентации $Q_1$ вокруг оси $n$ на угол $\alpha$. Также для ориентаций $Q_1$ и $Q_2$ задана хотя бы одна из
дуг-касательных $N_1(u)$ и $N_2(u)$, как было описано в предыдущем подразделе. Дуги-касательные будут результатами
поворотов из ориентаций $Q_1$ и $Q_2$ вокруг осей $m_1$ и $m_2$ на углы $\psi_1$ и $\psi_2$.

Таким образом:

$$
P(u)=R(n,u\alpha)Q_1, \quad N_1(u)=R(m_1,u\psi_1)Q_1, \quad N_2(u)=R(m_2,u\psi_2)p_2.
$$

Если задана дуга $N_2(u)$, введём дополнительное обозначение:

$$
N_3(u)=N_2(1-u), \quad u \in [0,1].
$$

Требуется провести на поверхности сферы $S^3$ кривую $R(u)$, соединяющую ориентации $Q_1$ и $Q_2$ и удовлетворяющую
следующим требованиям:

\begin{equation*}
R^{(k)}(0)=
  \begin{cases}
    N_1^{(k)}(0), & \text{если $N_1(u)$ задана}, \\
    P^{(k)}(0),   & \text{иначе};
  \end{cases}
\end{equation*}

\begin{equation*}
R^{(k)}(1)=
  \begin{cases}
    N_3^{(k)}(1), & \text{если $N_2(u)$ задана}, \\
    Q^{(k)}(1),   & \text{иначе};
  \end{cases}
\end{equation*}

$$
k \in {0,1,\dots,l}, \quad l \in \mathbb{N}.
$$

Рассмотрим три возможных случая.

\bigskip
\textit{Случай 1.}

Пусть угол $\phi_1$ задан, а угол $\phi_2$ "--- нет.

Тогда следует только построить деформацию из кривой $N_1(u)$ в кривую $P(u)$. Итоговая кривая $R(u)$ может быть
выражена следующей формулой:

$$
R(u)=R(n,w_l(u)u\alpha)R(m_1,(1-w_l(u))u\psi_1)Q_1, \quad u \in [0,1].
$$

Отсюда выразим функцию $R_1(u)$, описывающую для каждого $u$ некоторый поворот из ориентации $Q_1$:

\begin{equation}
R_1(u)=R(n,w_l(u)u\alpha)R(m_1,(1-w_l(u))u\psi_1), \quad u \in [0,1].
\label{orientation-first-deformed-curve}
\end{equation}

\bigskip
\textit{Случай 2.}

Пусть угол $\phi_2$ задан, а угол $\phi_1$ "--- нет.

Тогда следует построить вытягивание поворота $R(n,u\alpha)$ посредством поворота $R(m_2,u\psi_2)$. Итоговая кривая
$R(u)$ будет выражена следующей формулой:

$$
R(u)=R(m_2,(1-w_l(1-u))(1-u)\psi_2)R(n,w_l(u)u\alpha)Q_1, \quad u \in [0,1].
$$

Отсюда выразим функцию $R_2(u)$, описывающую для каждого $u$ некоторый поворот из ориентации $Q_1$:

\begin{equation}
R_2(u)=R(m_2,(1-w_l(1-u))(1-u)\psi_2)R(n,w_l(u)u\alpha), \quad u \in [0,1].
\label{orientation-second-deformed-curve}
\end{equation}

\bigskip
\textit{Случай 3.}

Пусть теперь заданы оба угла $\phi_1$ и $\phi_2$.

Вначале вычислим функции $R_1(u)$ и $R_2(u)$ по формулам (\ref{orientation-first-deformed-curve}) и
(\ref{orientation-second-deformed-curve}). Далее необходимо лишь построить деформацию $R_1(u)$ в $R_2(u)$:

$$
R_0(u)=R_2(w_l(u)u)R_1((1-w_l(u))u).
$$

Итовая кривая $R(u)$ имеет вид:

$$
R(u)=R_0(u)Q_1.
$$

\subsection*{Алгоритм построения кривой}

Приведём теперь детальное описание алгоритма построения кривой $R(t)$, проходящей последовательно через ориентации
$Q_1,\dots,Q_k$ под углами $\phi_1,\dots,\phi_k$.

\bigskip
\textit{Шаг} 0.

\begin{enumerate}
\item Построим малые дуги через каждые три последовательные ориентации $Q_i$, $Q_{i+1}$, $Q_{i+2}$,
$i \in \{1,\dots,k-2\}$, используя формулу (\ref{orientation-arc}), вычислим кривые $S_i(u)$ и $V_i(u)$,
$i \in \{1,\dots,k-2\}$, как участки малых дуг от $Q_i$ до $Q_{i+1}$ и от $Q_{i+1}$ до $Q_{i+2}$. Запомним углы
соответствующих дуг, как $\psi_i$ и $\chi_i$.
\item Построим большие дуги, соединяющие каждые две последовательные ориентации $Q_i$, $Q_{i+1}$, $i \in \{1,\dots,k-1\}$,
вычислим кривые $Z_i(u)$ по формуле (\ref{orientation-big-arc}), запомним при этом угол поворота дуги, как $\alpha_i$.
\item Построим дуги-касательные для тех ориентаций $Q_i$, $i \in \{1,\dots,k\}$, для которых задан угол $\phi_i$.
Вычислим дугу $N_i^+(u)$ для угла $\phi_i$ (прямое направление касательной), как результат поворота из ориентации $Q_i$
на угол $\alpha_i$, и дугу $N_i^-(u)$ для угла $\phi_i+\pi$ (обратное направление касательной), как результат поворота
из ориентации $Q_i$ на угол $-\alpha_{i-1}$.
\end{enumerate}

\bigskip
\textit{Шаг} 1.

Определим первый сегмент $R_1(t)$ кривой $R(t)$.

Если не заданы углы $\phi_1$ и $\phi_2$, достаточно определить $R_1(t)$ по формуле:

$$
R_1(t)=S_1(u(t)).
$$

Если задан хотя бы один из углов, следует использовать формулы из предыдущего подраздела. Применить их следует к
ориентациям $Q_1$ и $Q_2$, дугам-касательным $N_1^+(u)$ и $N_2^-(u)$, а в качестве кривой, соединяющей точки, нужно
выбрать либо $S_1(u)$, если не задан $\phi_2$, либо $Z_1(u)$ в ином случае, вместе с соответствующими углами
поворотов и осями. После применения формулы в полученной записи нужно заменить $u$ на $u(t)$.

В итоговой записи $u(t)$ имеет следующее значение:

$$
u(t)=\frac{t-t_1}{t_2-t_1}, \quad t \in [t_1,t_2].
$$

Значения $t_1$ и $t_2$ следует выбирать, исходя из каких-либо дополнительных кинематических требований или ограничений.

\bigskip
\textit{Шаг} $i$, где $i \in \{2,3,\dots,k-2\}$.

Вычислим сегмент $R_i(t)$ кривой $R(t)$.

Если не заданы углы $\phi_i$ и $\phi_{i+1}$, достаточно определить $R_i(t)$, как деформацию $V_{i-1}(u)$ в $S_i(u)$.
Если задан хотя бы один из углов, следует использовать формулы из предыдущего подраздела. Применить их следует к
ориентациям $Q_i$ и $Q_{i+1}$, дугам-касательным $N_i^+(u)$ и $N_{i+1}^-(u)$, а в качестве кривой, соединяющей точки,
нужно выбрать либо $S_i(u)$, если не задан $\phi_{i+1}$, либо $V_{i-1}(u)$, если не задан $\phi_i$, либо $Z_i(u)$
в ином случае, вместе с соответствующими углами поворотов и осями. После применения формулы в полученной записи нужно
заменить $u$ на $u(t)$.

В итоговой записи $u(t)$ имеет следующее значение:

$$
u(t)=\frac{t-t_i}{t_{i+1}-t_i}, \quad t \in [t_i,t_{i+1}].
$$

Чтобы обеспечить параметрическую непрерывность кривой $R(t)$ в ориентациях $Q_i$, значения $t_i$ следует выбирать
следующим образом:

$$
t_{i+1}=t_i+\frac{t_i-t_{i-1}}{\psi_{i-1}}\chi_{i-1}, \quad i \in \{2,3,\dots,k-2\}.
$$

\bigskip
\textit{Шаг} $(k-1)$.

Вычислим сегмент $R_{k-1}(t)$ кривой $R(t)$.

Если не заданы углы $\phi_{k-1}$ и $\phi_k$, достаточно определить $R_{k-1}(t)$ по формуле:

$$
R_{k-1}(t)=V_{k-2}(u(t)).
$$

Если задан хотя бы один из углов, следует использовать формулы из предыдущего подраздела. Применить их следует к
ориентациям $Q_{k-1}$ и $Q_k$, дугам-касательным $N_{k-1}^+(u)$ и $N_k^-(u)$, а в качестве кривой, соединяющей точки,
нужно выбрать либо $V_{k-2}(u)$, если не задан $\phi_{k-1}$, либо $Z_{k-1}(u)$ в ином случае, вместе с соответствующими
углами поворотов и осями. После применения формулы в полученной записи нужно заменить $u$ на $u(t)$.

В итоговой записи $u(t)$ имеет следующее значение:

$$
u(t)=\frac{t-t_k}{t_{k+1}-t_k}, \quad t \in [t_k,t_{k+1}].
$$

Чтобы обеспечить параметрическую непрерывность кривой $R(t)$ в ориентации $Q_{k-1}$, значение $t_{k+1}$ следует выбирать
следующим образом:

$$
t_k=t_{k-1}+\frac{t_{k-1}-t_{k-2}}{\psi_{k-2}}\chi_{k-2}.
$$
