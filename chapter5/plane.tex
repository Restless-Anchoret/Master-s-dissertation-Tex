\section{На плоскости}

Рассмотрим задачу построения кривой по набору точек и направлениям касательных на плоскости. Дан набор $k$ точек
$\{p_i\}$ в $\mathbb{R}^2$ и $k$ углов $\{\varphi_i\}$, причём некоторые $\varphi_i$ могут быть не заданы. Нужно построить
непрерывно-дифференцируемую до заданного порядка кривую, проходящую последовательно через все эти точки под заданными
углами к оси $Ox$.

\subsection*{Построение отрезка-касательной}

Поговорим вначале о том, как построить отрезок в направлении касательной под заданным углом.

Пусть дана точка $p \in \mathbb{R}^2$, угол $\varphi>0$ и некоторое число $L>0$. Нужно построить отрезок из точки $p$
в направлении, заданном углом $\varphi$ к оси $Ox$, причём отрезок должен иметь длину $L$.

Вычислим точку $q$, в которой должен заканчиваться искомый отрезок. Её можно найти по следующей формуле:

$$
q=p+(L\cos\varphi,L\sin\varphi).
$$

После этого остаётся вычислить $p(u)$, где $u \in [0,1]$, по формуле (\ref{plane-segment}) из предыдущей главы.

\subsection*{Построение кривой с заданными направлениями \mbox{касательных} в точках}

Пусть теперь есть две различные точки $p_1$ и $p_2$, соединённые некоторой непрерыв\-но-дифферецируемой до любого
порядка кривой $p(u)$, и для этих точек задан хотя бы один из отрезков-касательных $q_1(u)$ и $q_2(u)$, как было описано
в предыдущем подразделе.

Если задан отрезок $q_2(u)$, введём дополнительное обозначение:

$$
q_3(u)=q_2(1-u), \quad u \in [0,1].
$$

Требуется провести кривую $r(t)$, соединяющую точки $p_1$ и $p_2$ и удовлетворяющую следующим требованиям:

\begin{equation*}
r^{(k)}(0)=
  \begin{cases}
    q_1^{(k)}(0), & \text{если $q_1(u)$ задан}, \\
    p^{(k)}(0),   & \text{иначе};
  \end{cases}
\end{equation*}

\begin{equation*}
r^{(k)}(1)=
  \begin{cases}
    q_3^{(k)}(1), & \text{если $q_2(u)$ задан}, \\
    p^{(k)}(1),   & \text{иначе};
  \end{cases}
\end{equation*}

$$
k \in {0,1,\dots,l}, \quad l \in \mathbb{N}.
$$

Рассмотрим три возможных случая.

\bigskip
\textit{Случай 1.}

Пусть угол $\varphi_1$ задан, а угол $\varphi_2$ "--- нет. Тогда следует только построить деформацию из кривой $q_1(u)$ в
кривую $p(u)$. Для этого введём обозначения:

$$
\tilde q_1(u)=q_1(u)-p_1, \quad \tilde p(u)=p(u)-p_1, \quad u \in [0,1].
$$

Теперь итоговую кривую можно выразить по следующей формуле:

\begin{equation}
r(u)=\tilde p(\omega_l(u)u)+\tilde q_1((1-\omega_l(u))u)+p_1, \quad u \in [0,1].
\label{plane-first-deformed-curve}
\end{equation}

\bigskip
\textit{Случай 2.}

Пусть угол $\varphi_2$ задан, а угол $\varphi_1$ "--- нет. Введём дополнительные обозначения:

$$
\tilde q_2(u)=q_2(u)-p_2, \quad \tilde p(u)=p(u)-p_1, \quad u \in [0,1].
$$

Теперь следует построить вытягивание кривой $\tilde p(u)$ посредством кривой $\tilde q_2(u)$, после чего получить
итоговую кривую $r(t)$:

\begin{equation}
r(u)=\tilde q_2((1-\omega_l(1-u))(1-u))+\tilde p(\omega_l(u)u)+p_1, \quad u \in [0,1].
\label{plane-second-deformed-curve}
\end{equation}

\bigskip
\textit{Случай 3.}

Пусть теперь заданы оба угла $\varphi_1$ и $\varphi_2$.

Вначале построить две кривые $r_1(u)$ и $r_2(u)$ по формулам, соответственно, (\ref{plane-first-deformed-curve}),
(\ref{plane-second-deformed-curve}) и введём дополнительные обозначения:

$$
\tilde r_1(u)=r_1(u)-p_1, \quad \tilde r_2(u)=r_2(u)-p_1, \quad u \in [0,1].
$$

Далее построим деформацию $\tilde r_1(u)$ в $\tilde r_2(u)$:

\begin{equation}
r(u)=\tilde r_2(\omega_l(u)u)+\tilde r_1((1-\omega_l(u))u)+p_1, \quad u \in [0,1].
\label{plane-result-deformed-curve}
\end{equation}

\subsection*{Алгоритм построения кривой}

Приведём теперь детальное описание алгоритма построения кривой $r(t)$, проходящей последовательно через точки
$p_1,\dots,p_k$ под углами $\varphi_1,\dots,\varphi_k$.

\bigskip
\textit{Шаг} 0.

\begin{enumerate}
\item Построим окружности через каждые три последовательные точки $p_i$, $p_{i+1}$, $p_{i+2}$, $i \in \{1,\dots,k-2\}$,
вычислим кривые $s_i(u)$ и $v_i(u)$, $i \in \{1,\dots,k-2\}$ по формулам (\ref{arc-phi}) и (\ref{arc-psi}), запомним
углы соответствующих дуг, как $\psi_i$ и $\chi_i$.
\item Построим отрезки, соединяющие каждые две последовательные точки $p_i$, $p_{i+1}$, $i \in \{1,\dots,k-1\}$,
вычислим кривые $z_i(u)$ по формуле (\ref{plane-segment}).
\item Построим отрезки-касательные для тех точек $p_i$, $i \in \{1,\dots,k\}$, для которых задан угол $\varphi_i$.
Вычислим отрезок $q_i^+(u)$ для угла $\varphi_i$ (прямое направление касательной) и отрезок $q_i^-(u)$ для угла
$\varphi_i+\pi$ (обратное направление касательной). Длины касательных $L_i$ можно либо задавать произвольные и одинаковые,
либо использовать эти значения для корректировки итоговой кривой.
\end{enumerate}

\bigskip
\textit{Шаг} 1.

Определим первый сегмент $r_1(t)$ кривой $r(t)$.

Если не заданы углы $\varphi_1$ и $\varphi_2$, достаточно определить $r_1(t)$ по формуле:

$$
r_1(t)=s_1(u(t)).
$$

Если задан хотя бы один из углов, следует использовать формулы из предыдущего подраздела. Применить их следует к
точкам $p_1$ и $p_2$, отрезкам-касательным $q_1^+(u)$ и $q_2^-(u)$, а в качестве кривой, соединяющей точки, нужно
выбрать либо $s_1(u)$, если не задан $\varphi_2$, либо $z_1(u)$ в ином случае. После применения формулы в полученной
записи нужно заменить $u$ на $u(t)$.

В итоговой записи $u(t)$ имеет следующее значение:

$$
u(t)=\frac{t-t_1}{t_2-t_1}, \quad t \in [t_1,t_2].
$$

Значения $t_1$ и $t_2$ следует выбирать, исходя из каких-либо дополнительных кинематических требований или ограничений.

\bigskip
\textit{Шаг} $i$, где $i \in \{2,3,\dots,k-2\}$.

Вычислим сегмент $r_i(t)$ кривой $r(t)$.

Если не заданы углы $\varphi_i$ и $\varphi_{i+1}$, достаточно определить $r_i(t)$, как деформацию $v_{i-1}(u)$ в $s_i(u)$.
Если задан хотя бы один из углов, следует использовать формулы из предыдущего подраздела. Применить их следует к
точкам $p_i$ и $p_{i+1}$, отрезкам-касательным $q_i^+(u)$ и $q_{i+1}^-(u)$, а в качестве кривой, соединяющей точки,
нужно выбрать либо $s_i(u)$, если не задан $\varphi_{i+1}$, либо $v_{i-1}(u)$, если не задан $\varphi_i$, либо $z_i(u)$
в ином случае. После применения формулы в полученной записи нужно заменить $u$ на $u(t)$.

В итоговой записи $u(t)$ имеет следующее значение:

$$
u(t)=\frac{t-t_i}{t_{i+1}-t_i}, \quad t \in [t_i,t_{i+1}].
$$

Чтобы обеспечить параметрическую непрерывность кривой $r(t)$ в точках $p_i$, значения $t_i$ следует выбирать
следующим образом:

$$
t_{i+1}=t_i+\frac{t_i-t_{i-1}}{|\psi_{i-1}|}|\chi_{i-1}|, \quad i \in \{2,3,\dots,k-2\}.
$$

\bigskip
\textit{Шаг} $(k-1)$.

Вычислим сегмент $r_{k-1}(t)$ кривой $r(t)$.

Если не заданы углы $\varphi_{k-1}$ и $\varphi_k$, достаточно определить $r_{k-1}(t)$ по формуле:

$$
r_{k-1}(t)=v_{k-2}(u(t)).
$$

Если задан хотя бы один из углов, следует использовать формулы из предыдущего подраздела. Применить их следует к
точкам $p_{k-1}$ и $p_k$, отрезкам-касательным $q_{k-1}^+(u)$ и $q_k^-(u)$, а в качестве кривой, соединяющей точки,
нужно выбрать либо $v_{k-2}(u)$, если не задан $\varphi_{k-1}$, либо $z_{k-1}(u)$ в ином случае. После применения формулы
в полученной записи нужно заменить $u$ на $u(t)$.

В итоговой записи $u(t)$ имеет следующее значение:

$$
u(t)=\frac{t-t_k}{t_{k+1}-t_k}, \quad t \in [t_k,t_{k+1}].
$$

Чтобы обеспечить параметрическую непрерывность кривой $r(t)$ в точке $p_{k-1}$, значение $t_{k+1}$ следует выбирать
следующим образом:

$$
t_k=t_{k-1}+\frac{t_{k-1}-t_{k-2}}{|\psi_{k-2}|}|\chi_{k-2}|.
$$
