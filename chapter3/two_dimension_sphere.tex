\section{На двумерной сфере}

Рассмотрим аналогичную задачу на двумерной сфере. Дан набор точек ${p_i}$, лежащих на поверхности сферы $S^2$, и
нужно построить непрерывно-дифференцируемую кривую до заданного порядка, проходящую последовательно через все эти
точки и лежащую на поверхности $S^2$.

\subsection*{Построение малой дуги}

Рассмотрим вначале некоторые три различные точки $p_1, p_2, p_3$, лежащие на поверхности $S^2$, и поговорим о
том, как построить малую дугу, проходящую последовательно через эти три точки. Дуга будет являться результатом поворота
точки $p_1$ вокруг некоторой оси $n$ вначале на угол $\phi$ до точки $p_2$, и затем на угол $\psi$ до точки $p_3$.
Точки $p_1, p_2, p_3$ лежат на малой окружности, и центр этой окружности мы обозначим $c$.

Можно показать, что ось $n$ и центр малой окружности $c$ выражаются следующими формулами:

$$
n=\frac{(p_3-p_2)\times(p_1-p_2)}{|(p_3-p_2)\times(p_1-p_2)|},
$$

$$
c=\frac{[p_1p_2p_3]}{|(p_3-p_2)\times(p_1-p_2)|}n.
$$

Далее определим углы $\phi$ и $\psi$:

$$
\phi=\angle P_1CP_2, \quad \psi=\angle P_2CP_3.
$$

Для краткости введём дополнительные обозначения:

$$
q_1=p_1-c, \quad q_2=p_2-c, \quad q_3=p_3=c,
$$

$$
r_1=|q_1\times q_2|, \quad r_2=|q_2\times q_3|,
$$

$$
s_1=q_1\cdot q_2, \quad s_2=q_2\cdot q_3.
$$

Тогда углы $\phi$ и $\psi$ можно определить по следующим формулам:

\begin{equation*}
\phi=
 \begin{cases}
   \arctantwo(r_1,s_1),      & (q_1\times q_2)\cdot n>0, \\
   2\pi-\arctantwo(r_1,s_1), & (q_1\times q_2)\cdot n<0;
 \end{cases}
\end{equation*}

\begin{equation*}
\phi=
 \begin{cases}
   \arctantwo(r_2,s_2),      & (q_2\times q_3)\cdot n>0, \\
   2\pi-\arctantwo(r_2,s_2), & (q_2\times q_3)\cdot n<0.
 \end{cases}
\end{equation*}

Теперь требуемая дуга малой окружности может быть описана следующей формулой:

$$
p(u)=R(n,u(\phi+\psi))p_1, \quad u \in [0,1].
$$

Кроме того, малые дуги от $p_1$ к $p_2$ и от $p_2$ к $p_3$ можно выразить следующим образом:

$$
f_1(u)=R(n,u\phi)p_1, \quad f_2(u)=R(n,u\psi)p_2, \quad u \in [0,1].
$$

\subsection*{Деформация малых дуг}

Теперь рассмотрим две различные точки $p_1$ и $p_2$, соединённые двумя различными малыми дугами. Пусть первая дуга
$p(u)$ будет результатом поворота точки $p_1$ вокруг оси $m$ на угол $\phi$, а вторая дуга $q(u)$ "--- результатом
поворота точки $p_1$ вокруг оси $n$ на угол $\psi$:

$$
p(u)=R(m,u\phi)p_1, \quad q(u)=R(n,u\psi)p_1, \quad u \in [0,1].
$$

Требуется построить новую кривую $r(u)$, удовлетворяющую следующим граничным условиям:

$$
r^{(k)}(0)=p^{(k)}(0), \quad r^{(k)}(1)=q^{(k)}(1),
$$

$$
k \in {0,1,\dots,l}, \quad l \in \mathbb{N}.
$$

В этом месте следует применить теорему о деформации кривых. Собственно ортогональные матрицы, представляющие повороты
точки $p_1$, принадлежат мультипликативной группе с операцией умножения матриц. Имеем две кривые в пространстве
собственно ортогональных матриц: $R(m,u\phi)$ и $R(n,u\psi)$. При $u=0$ они равняются единичной матрице, поэтому эти
кривые выходят из единицы мультипликативной группы. Значит, к ним можно применить теорему о деформации кривых.

Таким образом, можно показать, что искомая кривая $r(u)$ представима следующим образом:

$$
r(u)=R(n,w_l(u)u\psi)R(m,(1-w_l(u))u\phi)p_1, \quad u \in [0,1].
$$

\subsection*{Алгоритм построения кривой}

Приведём теперь детальное описание алгоритма решения исходной задачи.
