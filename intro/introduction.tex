\chapter*{Введение}
\addcontentsline{toc}{chapter}{Введение}

Методы построения сплайн-кривых находят применение в различных областях: например, в 3D-графике для построения гладкой
анимации объектов и в робототехнике для построения гладкой траектории движения руки робота.

В рамках данной работы будут исследованы методы построения сплайн-кривых, предложенные известным математиком
А.П.~Побегайло \cite{pobegaylo}, и предложены новые методы, которые решают задачи с дополнительными условиями,
налагаемыми на сплайн-кривые.

А.П.~Побегайло в своей монографии описывает теорию построения сплайн-кривых, основанную на
использовании свойств полиномов Бернштейна и сглаживающих полиномов, а также на использовании теорем о деформации и
сглаживании кривых. Он рассматривает алгоритмы построения кривой по набору точек, а также кривой Безье применительно
к двумерной и ориентационной сферам. Характерной особенностью предложенных им алгоритмов является то, что получаемая
в результате кривая является непрерывно-дифференцируемой до произвольного порядка.

В данной работе теория, предложенная А.П.~Побегайло, применяется к плоскости, а также используется для
решения задачи построения сплайн-кривой по набору точек и направлениям касательных в точках на двумерной и
ориентационной сферах. Также все рассмотренные алгоритмы построения кривых реализованы программно в виде
приложения с 3D-визуализацией, в котором будет использован самописный графический движок.

Можно надеяться, что данная работа будет полезна в областях компьютерной графики и робототехники.
