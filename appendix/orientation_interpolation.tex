\chapter{Интерполирование кривой на~ориентационной сфере}

\section*{\texttt{AbstractOrientationCurveCreator.java}}
\begin{verbatim}
package com.ran.engine.factories.interpolation.curvecreators;

// Импорт классов
// ...

public abstract class AbstractOrientationCurveCreator extends
    AbstractInterpolatedCurveCreator<
        Quaternion, Quaternion, SimpleInputParameters> {

    @Override
    protected void validateVerticesList(List<Quaternion> verticesList) {
        super.validateVerticesList(verticesList);
        if (verticesList.stream().anyMatch(quaternion -> !quaternion.isIdentity())) {
            throw new InterpolationException("All quaternions must be identity");
        }
    }

    protected DoubleFunction<Quaternion> buildFinalCurve(
            List<Double> timeMoments,
            List<Quaternion> quaternions,
            List<DoubleFunction<Quaternion>> rotationsOnSegments,
            int segmentsQuantity) {
        TimeMomentsUtil timeMomentsUtil = TimeMomentsUtil.getInstance();
        List<DoubleFunction<Quaternion>> orientationCurveSegments =
            new ArrayList<>(segmentsQuantity);
        for (int i = 0; i < segmentsQuantity; i++) {
            double startTime = timeMoments.get(i);
            double endTime = timeMoments.get(i + 1);
            DoubleFunction<Quaternion> currentRotation = rotationsOnSegments.get(i);
            Quaternion currentQuaternion = quaternions.get(i);
            DoubleFunction<Quaternion> curveSegmentWithoutAligning =
                new DoubleFunction<>(
                    point -> currentRotation.apply(point)
                        .multiply(currentQuaternion), 0.0, 1.0
            );
            DoubleFunction<Quaternion> alignedCurveSegment =
                    curveSegmentWithoutAligning.superposition(timeMomentsUtil
                        .buildAligningFunction(startTime, endTime));
            orientationCurveSegments.add(alignedCurveSegment);
        }
        return DoubleMultifunction.makeMultifunction(orientationCurveSegments);
    }
}
\end{verbatim}

\section*{\texttt{OrientationArcsBuilder.java}}
\begin{verbatim}
package com.ran.engine.factories.interpolation.tools;

// Импорт классов
// ...

public class OrientationArcsBuilder {

    private static final OrientationArcsBuilder INSTANCE =
        new OrientationArcsBuilder();

    public static OrientationArcsBuilder getInstance() {
        return INSTANCE;
    }

    public Result buildArcsBetweenQuaternionsOnThreeDimensionalSphere(
            Quaternion p1, Quaternion p2, Quaternion p3) {
        Quaternion hNotNormalized = p1.quaternionVectorMultiply(p2, p3);
        if (ArithmeticOperations.doubleEquals(hNotNormalized.getNorm(), 0.0)) {
            OrientationBigArcsBuilder bigArcsBuilder =
                OrientationBigArcsBuilder.getInstance();
            OrientationBigArcsBuilder.Result firstArc = bigArcsBuilder
                .buildOrientationBigArcsBetweenQuaternions(p1, p2);
            OrientationBigArcsBuilder.Result secondArc = bigArcsBuilder
                .buildOrientationBigArcsBetweenQuaternions(p2, p3);
            return new Result(firstArc.getRotation(), secondArc.getRotation(),
                firstArc.getAngle(), secondArc.getAngle());
        } else {
            Quaternion h = hNotNormalized.normalized();
            Quaternion hConjugate = h.getConjugate();

            Quaternion r1 = p1.multiply(hConjugate);
            Quaternion r2 = p2.multiply(hConjugate);
            Quaternion r3 = p3.multiply(hConjugate);

            ThreeDoubleVector m1 = r1.getVector();
            ThreeDoubleVector m2 = r2.getVector();
            ThreeDoubleVector m3 = r3.getVector();

            ArcsBuilder.Result arcsBuilderResult = ArcsBuilder.getInstance()
                    .buildArcsBetweenVerticesOnSphere(m1, m2, m3);

            DoubleFunction<Quaternion> firstRotation = buildRotationFunction(
                    arcsBuilderResult.getFirstRotation(), m1, r1);
            DoubleFunction<Quaternion> secondRotation = buildRotationFunction(
                    arcsBuilderResult.getSecondRotation(), m2, r2);

            return new Result(firstRotation, secondRotation,
                    arcsBuilderResult.getFirstAngle(),
                    arcsBuilderResult.getSecondAngle());
        }
    }

    private DoubleFunction<Quaternion> buildRotationFunction(
            DoubleFunction<DoubleMatrix> matrixRotation,
            ThreeDoubleVector m, Quaternion r) {
        return new DoubleFunction<>(
                point -> {
                    DoubleMatrix rotation = matrixRotation.apply(point);
                    Quaternion leftFactor = Quaternion.createFromVector(
                            new ThreeDoubleVector(rotation.multiply(
                                m.getDoubleVector())));
                    return leftFactor.multiply(r.getConjugate());
                },
                0.0, 1.0
        );
    }

    public static class Result {
        private final DoubleFunction<Quaternion> firstRotation;
        private final DoubleFunction<Quaternion> secondRotation;
        private final double firstAngle;
        private final double secondAngle;

        // Конструкторы, get- и set-методы
    }
}
\end{verbatim}

\section*{\texttt{OrientationByPointsCurveCreator.java}}
\begin{verbatim}
package com.ran.engine.factories.interpolation.curvecreators;

// Импорт классов
// ...

public class OrientationByPointsCurveCreator extends
    AbstractOrientationCurveCreator {

    private static final OrientationByPointsCurveCreator INSTANCE =
        new OrientationByPointsCurveCreator();

    public static OrientationByPointsCurveCreator getInstance() {
        return INSTANCE;
    }

    @Override
    public DoubleFunction<Quaternion> interpolateCurve(
            List<Quaternion> quaternions,
            SimpleInputParameters parameters,
            int degree) {
        validateVerticesList(quaternions);

        double t0 = parameters.getT0();
        double t1 = parameters.getT1();
        int k = quaternions.size();

        CurvesDeformationCreator deformationCreator =
            CurvesDeformationCreator.getInstance();
        OrientationArcsBuilder orientationArcsBuilder =
            OrientationArcsBuilder.getInstance();
        TimeMomentsUtil timeMomentsUtil = TimeMomentsUtil.getInstance();

        List<DoubleFunction<Quaternion>> rotationsOnSegments =
            new ArrayList<>(k - 1);
        List<Pair<Double, Double>> rotationAngles = new ArrayList<>(k - 2);
        OrientationArcsBuilder.Result currentOrientationArcsBuildingResult =
                orientationArcsBuilder
                    .buildArcsBetweenQuaternionsOnThreeDimensionalSphere(
                        quaternions.get(0), quaternions.get(1), quaternions.get(2));
        rotationsOnSegments.add(currentOrientationArcsBuildingResult
            .getFirstRotation());
        rotationAngles.add(currentOrientationArcsBuildingResult.getAngles());

        for (int i = 1; i < k - 2; i++) {
            OrientationArcsBuilder.Result nextOrientationArcsBuildingResult =
                    orientationArcsBuilder
                        .buildArcsBetweenQuaternionsOnThreeDimensionalSphere(
                            quaternions.get(i), quaternions.get(i + 1),
                            quaternions.get(i + 2));
            DoubleFunction<Quaternion> deformedFunction =
                deformationCreator.deformCurves(
                    currentOrientationArcsBuildingResult.getSecondRotation(),
                    nextOrientationArcsBuildingResult.getFirstRotation(), degree);
            rotationsOnSegments.add(deformedFunction);
            rotationAngles.add(nextOrientationArcsBuildingResult.getAngles());
            currentOrientationArcsBuildingResult = nextOrientationArcsBuildingResult;
        }
        rotationsOnSegments.add(currentOrientationArcsBuildingResult
            .getSecondRotation());

        List<Double> timeMoments = timeMomentsUtil.countTimeMoments(
            rotationAngles, t0, t1, k);
        return buildFinalCurve(timeMoments, quaternions, rotationsOnSegments, k - 1);
    }
}
\end{verbatim}

\section*{\texttt{OrientationBigArcsBuilder.java}}
\begin{verbatim}
package com.ran.engine.factories.interpolation.tools;

// Импорт классов
// ...

public class OrientationBigArcsBuilder {

    private static OrientationBigArcsBuilder INSTANCE =
        new OrientationBigArcsBuilder();

    public static OrientationBigArcsBuilder getInstance() {
        return INSTANCE;
    }

    public Result buildOrientationBigArcsBetweenQuaternions(
            Quaternion p1, Quaternion p2) {
        Quaternion r = p2.multiply(p1.getConjugate());
        ThreeDoubleVector axis = r.getVector().normalized();
        double cos = r.getScalar();
        double angle = (Math.acos(cos) * 2.0 + Math.PI) \% (2.0 * Math.PI) - Math.PI;
        DoubleFunction<Quaternion> rotation = new DoubleFunction<>(
                point -> Quaternion.createForRotation(axis, angle * point),
                0.0, 1.0
        );
        return new Result(rotation, angle);
    }

    public static class Result {
        private DoubleFunction<Quaternion> rotation;
        private double angle;

        // Конструкторы, get- и set-методы
    }
}
\end{verbatim}

\section*{\texttt{OrientationBezierCurveCreator.java}}
\begin{verbatim}
package com.ran.engine.factories.interpolation.curvecreators;

// Импорт классов
// ...

public class OrientationBezierCurveCreator extends AbstractOrientationCurveCreator {

    @Override
    public DoubleFunction<Quaternion> interpolateCurve(
            List<Quaternion> quaternions,
            SimpleInputParameters parameters, int degree) {
        validateVerticesList(quaternions);

        double t0 = parameters.getT0();
        double t1 = parameters.getT1();
        int k = quaternions.size();

        CurvesSmoothingCreator curvesSmoothingCreator =
            CurvesSmoothingCreator.getInstance();
        OrientationBigArcsBuilder bigArcsBuilder =
            OrientationBigArcsBuilder.getInstance();

        List<Quaternion> arcsCenters = new ArrayList<>(k - 1);
        for (int i = 0; i < k - 1; i++) {
            OrientationBigArcsBuilder.Result bigArcsBuilderResult = bigArcsBuilder
                    .buildOrientationBigArcsBetweenQuaternions(quaternions.get(i),
                        quaternions.get(i + 1));
            arcsCenters.add(bigArcsBuilderResult.getRotation().apply(0.5)
                .multiply(quaternions.get(i)));
        }

        List<DoubleFunction<Quaternion>> halfRotationsForward =
            new ArrayList<>(k - 1);
        List<DoubleFunction<Quaternion>> halfRotationsBack = new ArrayList<>(k - 1);
        for (int i = 0; i < k - 1; i++) {
            OrientationBigArcsBuilder.Result firstHalfBigArcsBuilderResult =
                bigArcsBuilder
                    .buildOrientationBigArcsBetweenQuaternions(quaternions.get(i),
                        arcsCenters.get(i));
            halfRotationsForward.add(firstHalfBigArcsBuilderResult.getRotation());
            OrientationBigArcsBuilder.Result secondHalfBigArcsBuilderResult =
                bigArcsBuilder
                    .buildOrientationBigArcsBetweenQuaternions(quaternions.get(i + 1),
                        arcsCenters.get(i));
            halfRotationsBack.add(secondHalfBigArcsBuilderResult.getRotation());
        }

        List<DoubleFunction<Quaternion>> smoothedRotations = new ArrayList<>(k);
        smoothedRotations.add(halfRotationsForward.get(0));
        for (int i = 1; i < k - 1; i++) {
            smoothedRotations.add(curvesSmoothingCreator.smoothCurves(
                    halfRotationsBack.get(i - 1),
                    halfRotationsForward.get(i), degree));
        }
        smoothedRotations.add(new DoubleFunction<>(
                point -> halfRotationsBack.get(k - 2).apply(1.0 - point),
                0.0, 1.0)
        );

        List<Double> timeMoments = new ArrayList<>(k + 1);
        double timeDelta = t1 - t0;
        for (int i = 0; i < k + 1; i++) {
            timeMoments.add(t0 + i * timeDelta);
        }
        return buildFinalCurve(timeMoments, quaternions, smoothedRotations, k);
    }
}
\end{verbatim}

\section*{\texttt{TangentOrientationBuilder.java}}
\begin{verbatim}
package com.ran.engine.factories.interpolation.tools;

// Импорт классов
// ...

public class TangentOrientationBuilder {

    private static final Logger LOG =
        LoggerFactory.getLogger(TangentOrientationBuilder.class);

    private static TangentOrientationBuilder INSTANCE =
        new TangentOrientationBuilder();
    private static final DoubleMatrix Z_HALF_PI_ROTATION =
        RotationCreator.getInstance().createRotation(
            ThreeDoubleVector.Z_ONE_THREE_DOUBLE_VECTOR, -Math.PI / 2.0);

    public static TangentOrientationBuilder getInstance() {
        return INSTANCE;
    }

    public Result buildTangentOrientation(Quaternion orientation,
                                          double tangentAngle,
                                          Double forwardRotationAngle,
                                          Double backRotationAngle) {
        LOG.trace("orientation = {}, tangentAngle = {}, forwardRotationAngle = {}, " +
            "backRotationAngle = {}",
                orientation, tangentAngle, forwardRotationAngle, backRotationAngle);
        ThreeDoubleVector a, b;
        ThreeDoubleVector point = orientation.multiply(Quaternion.createFromVector(
            ThreeDoubleVector.Z_ONE_THREE_DOUBLE_VECTOR))
                .multiply(orientation.getConjugate()).getVector();
        if (ArithmeticOperations.doubleEquals(point.getX(), 0.0) &&
                ArithmeticOperations.doubleEquals(point.getY(), 0.0)) {
            a = ThreeDoubleVector.X_ONE_THREE_DOUBLE_VECTOR;
            if (point.getZ() > 0.0) {
                b = ThreeDoubleVector.MINUS_Y_ONE_THREE_DOUBLE_VECTOR;
            } else {
                b = ThreeDoubleVector.Y_ONE_THREE_DOUBLE_VECTOR;
            }
        } else {
            a = new ThreeDoubleVector(Z_HALF_PI_ROTATION.multiply(
                    new ThreeDoubleVector(point.getX(), point.getY(), 0.0)
                        .getDoubleVector())).normalized();
            b = point.multiply(a).normalized();
        }
        ThreeDoubleVector n = a.multiply(Math.sin(tangentAngle))
            .add(b.multiply(-Math.cos(tangentAngle)));
        LOG.trace("a = {}, b = {}, n = {}", a, b, n);

        DoubleFunction<Quaternion> forwardRotation = null;
        DoubleFunction<Quaternion> backRotation = null;

        if (forwardRotationAngle != null) {
            forwardRotation = new DoubleFunction<>(
                    u -> Quaternion.createForRotation(n, -u * forwardRotationAngle),
                    0.0, 1.0);
        }
        if (backRotationAngle != null) {
            backRotation = new DoubleFunction<>(
                    u -> Quaternion.createForRotation(n, u * backRotationAngle),
                    0.0, 1.0);
        }
        return new Result(forwardRotation, backRotation,
            forwardRotationAngle, backRotationAngle);
    }

    public static class Result {
        private final DoubleFunction<Quaternion> forwardRotation;
        private final DoubleFunction<Quaternion> backRotation;
        private final Double forwardAngle, backAngle;

        // Конструкторы, get- и set-методы
    }
}
\end{verbatim}

\section*{\texttt{OrientationByTangentsCurveCreator.java}}
\begin{verbatim}
package com.ran.engine.factories.interpolation.curvecreators;

// Импорт классов
// ...

public class OrientationByTangentAnglesCurveCreator extends
    AbstractInterpolatedCurveCreator<
        Pair<Quaternion, Double>, Quaternion, SimpleInputParameters> {

    private static final Logger LOG = LoggerFactory.getLogger(
        OrientationByTangentAnglesCurveCreator.class);

    private static final OrientationByTangentAnglesCurveCreator INSTANCE =
        new OrientationByTangentAnglesCurveCreator();

    public static OrientationByTangentAnglesCurveCreator getInstance() {
        return INSTANCE;
    }

    @Override
    public DoubleFunction<Quaternion> interpolateCurve(
            List<Pair<Quaternion, Double>> quaternionsWithTangentAnglesList,
            SimpleInputParameters parameters, int degree) {
        validateVerticesList(quaternionsWithTangentAnglesList);

        double t0 = parameters.getT0();
        double t1 = parameters.getT1();
        int k = quaternionsWithTangentAnglesList.size();
        LOG.trace("t0 = {}, t1 = {}, k = {}", t0, t1, k);

        CurvesDeformationCreator deformationCreator =
            CurvesDeformationCreator.getInstance();
        OrientationArcsBuilder arcsBuilder = OrientationArcsBuilder.getInstance();
        OrientationBigArcsBuilder bigArcsBuilder =
            OrientationBigArcsBuilder.getInstance();
        TangentOrientationBuilder tangentBuilder =
            TangentOrientationBuilder.getInstance();
        TimeMomentsUtil timeMomentsUtil = TimeMomentsUtil.getInstance();

        LOG.trace("Before calling OrientationArcsBuilder");
        List<OrientationArcsBuilder.Result> smallArcsResults = new ArrayList<>(k - 2);
        for (int i = 1; i < k - 1; i++) {
            smallArcsResults.add(arcsBuilder
                .buildArcsBetweenQuaternionsOnThreeDimensionalSphere(
                    quaternionsWithTangentAnglesList.get(i - 1).getLeft(),
                    quaternionsWithTangentAnglesList.get(i).getLeft(),
                    quaternionsWithTangentAnglesList.get(i + 1).getLeft()
            ));
        }
        LOG.trace("smallArcsResults = {}", smallArcsResults);

        LOG.trace("Before calling OrientationBigArcsBuilder");
        List<OrientationBigArcsBuilder.Result> bigArcsResults =
            new ArrayList<>(k - 1);
        for (int i = 0; i < k - 1; i++) {
            bigArcsResults.add(bigArcsBuilder
                .buildOrientationBigArcsBetweenQuaternions(
                    quaternionsWithTangentAnglesList.get(i).getLeft(),
                    quaternionsWithTangentAnglesList.get(i + 1).getLeft()
            ));
        }

        List<TangentOrientationBuilder.Result> tangentBuilderResults =
            new ArrayList<>(k);
        for (int i = 0; i < k; i++) {
            LOG.trace("Point #{} = {}", i, quaternionsWithTangentAnglesList.get(i));
            if (quaternionsWithTangentAnglesList.get(i).getRight() == null) {
                LOG.trace("Tangle angle is null");
                tangentBuilderResults.add(null);
            } else {
                LOG.trace("Processing angles");
                Double forwardAngle = null, backAngle = null;
                if (i + 1 < k) {
                    if (quaternionsWithTangentAnglesList
                            .get(i + 1).getRight() != null || i + 1 == k - 1) {
                        forwardAngle = Math.abs(bigArcsResults.get(i).getAngle());
                    } else {
                        forwardAngle = Math.abs(smallArcsResults
                            .get(i).getFirstAngle());
                    }
                }
                if (i - 1 >= 0) {
                    if (quaternionsWithTangentAnglesList
                            .get(i - 1).getRight() != null || i - 1 == 0) {
                        backAngle = Math.abs(bigArcsResults.get(i - 1).getAngle());
                    } else {
                        backAngle = Math.abs(smallArcsResults
                            .get(i - 2).getSecondAngle());
                    }
                }
                tangentBuilderResults.add(tangentBuilder.buildTangentOrientation(
                        quaternionsWithTangentAnglesList.get(i).getLeft(),
                        quaternionsWithTangentAnglesList.get(i).getRight(),
                        forwardAngle,
                        backAngle
                ));
            }
        }

        List<Pair<Double, Double>> rotationAngles = new ArrayList<>(k - 2);
        for (int i = 0; i < k - 2; i++) {
            if (quaternionsWithTangentAnglesList.get(i + 1).getRight() == null) {
                rotationAngles.add(smallArcsResults.get(i).getAngles());
            } else {
                rotationAngles.add(tangentBuilderResults.get(i + 1).getAngles());
            }
        }
        LOG.trace("rotationAngles = {}", rotationAngles);

        LOG.trace("Before building rotationsOnSegments");
        List<DoubleFunction<Quaternion>> rotationsOnSegments = new ArrayList<>(k - 1);
        for (int i = 0; i < k - 1; i++) {
            LOG.trace("Building rotation between points {} and {}: {} and {}",
                    i, i + 1, quaternionsWithTangentAnglesList.get(i),
                    quaternionsWithTangentAnglesList.get(i + 1));
            if (quaternionsWithTangentAnglesList.get(i).getRight() != null &&
                    quaternionsWithTangentAnglesList.get(i + 1).getRight() != null) {
                LOG.trace("Angles are set on both points");
                DoubleFunction<Quaternion> firstDeformedCurve =
                    deformationCreator.deformCurves(
                        tangentBuilderResults.get(i).getForwardRotation(),
                        bigArcsResults.get(i).getRotation(), degree);
                DoubleFunction<Quaternion> secondDeformedCurve = deformationCreator
                    .deformCurvesWithCommonEnd(
                        bigArcsResults.get(i).getRotation(),
                        tangentBuilderResults.get(i + 1).getBackRotation(), degree);
                rotationsOnSegments.add(deformationCreator.deformCurves(
                        firstDeformedCurve, secondDeformedCurve, degree));
            } else if (quaternionsWithTangentAnglesList.get(i).getRight() != null) {
                LOG.trace("Angle is set only on the first point");
                if (i == k - 2) {
                    rotationsOnSegments.add(deformationCreator.deformCurves(
                            tangentBuilderResults.get(i).getForwardRotation(),
                            bigArcsResults.get(i).getRotation(), degree));
                } else {
                    rotationsOnSegments.add(deformationCreator.deformCurves(
                            tangentBuilderResults.get(i).getForwardRotation(),
                            smallArcsResults.get(i).getFirstRotation(), degree));
                }
            } else if (quaternionsWithTangentAnglesList
                    .get(i + 1).getRight() != null) {
                LOG.trace("Angle is set only on the second point");
                if (i == 0) {
                    rotationsOnSegments.add(deformationCreator
                        .deformCurvesWithCommonEnd(
                            bigArcsResults.get(i).getRotation(),
                            tangentBuilderResults.get(i + 1).getBackRotation(),
                            degree));
                } else {
                    rotationsOnSegments.add(deformationCreator
                        .deformCurvesWithCommonEnd(
                            smallArcsResults.get(i - 1).getSecondRotation(),
                            tangentBuilderResults.get(i + 1).getBackRotation(),
                            degree));
                }
            } else {
                LOG.trace("Angles are not set on both points");
                if (i == 0) {
                    rotationsOnSegments.add(smallArcsResults
                        .get(i).getFirstRotation());
                } else if (i == k - 2) {
                    rotationsOnSegments.add(smallArcsResults
                        .get(i - 1).getSecondRotation());
                } else {
                    rotationsOnSegments.add(deformationCreator.deformCurves(
                            smallArcsResults.get(i - 1).getSecondRotation(),
                            smallArcsResults.get(i).getFirstRotation(), degree));
                }
            }
        }

        List<Double> timeMoments = timeMomentsUtil
            .countTimeMoments(rotationAngles, t0, t1, k);
        LOG.trace("timeMoments = {}", timeMoments);

        LOG.trace("Before building curveSegments");
        List<DoubleFunction<Quaternion>> curveSegments = new ArrayList<>(k - 1);
        for (int i = 0; i < k - 1; i++) {
            LOG.trace("Building curve between {} and {} points");
            double startTime = timeMoments.get(i);
            double endTime = timeMoments.get(i + 1);
            DoubleFunction<Quaternion> currentRotation = rotationsOnSegments.get(i);
            Quaternion currentOrientation =
                quaternionsWithTangentAnglesList.get(i).getLeft();
            LOG.trace("startTime = {}, endTime = {}, currentOrientation = {}",
                startTime, endTime, currentOrientation);
            DoubleFunction<Quaternion> curveSegmentWithoutAligning =
                new DoubleFunction<>(
                    point -> currentRotation.apply(point)
                    .multiply(currentOrientation), 0.0, 1.0);
            DoubleFunction<Quaternion> alignedCurveSegment =
                    curveSegmentWithoutAligning.superposition(timeMomentsUtil
                        .buildAligningFunction(startTime, endTime));
            curveSegments.add(alignedCurveSegment);
        }
        return DoubleMultifunction.makeMultifunction(curveSegments);
    }

    @Override
    protected void validateVerticesList(
            List<Pair<Quaternion, Double>> quaternionsWithTangentAnglesList) {
        super.validateVerticesList(quaternionsWithTangentAnglesList);
        if (quaternionsWithTangentAnglesList.stream().anyMatch(
                quaternionWithTangentAngle -> !quaternionWithTangentAngle
                    .getLeft().isIdentity())) {
            throw new InterpolationException("All quaternions must be identity");
        }
    }
}
\end{verbatim}
