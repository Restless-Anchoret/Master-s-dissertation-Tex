\chapter{Вспомогательные классы}

\section*{\texttt{PolynomsCreator.java}}
\begin{verbatim}
package com.ran.engine.factories.interpolation.tools;

// Импорт классов
// ...

public class PolynomsCreator {

    private static final PolynomsCreator INSTANCE = new PolynomsCreator();

    public static PolynomsCreator getInstance() {
        return INSTANCE;
    }

    public DoubleFunction<SingleDouble> createBernsteinPolynom(int n, int m) {
        if (n < 0 || m < 0 || n < m) {
            throw new CreationException("Incorrect parameters while creating " +
                "Bernstein polynom");
        }
        long combinations = countCombinations(n, m);
        return new DoubleFunction<>(point -> new SingleDouble(
            combinations * power(1 - point, n - m) * power(point, m)), 0.0, 1.0);
    }

    public DoubleFunction<SingleDouble> createSmoothingPolynom(int k) {
        if (k < 0) {
            throw new CreationException("Incorrect parameters while creating " +
                "smoothing polynom");
        }
        List<DoubleFunction<SingleDouble>> bernsteinPolynoms = new ArrayList<>(k + 1);
        for (int i = k + 1; i <= 2 * k + 1; i++) {
            bernsteinPolynoms.add(createBernsteinPolynom(2 * k + 1, i));
        }
        return new DoubleFunction<>(point -> new SingleDouble(
                bernsteinPolynoms.stream()
                .mapToDouble(polynom -> polynom.apply(point).getValue()).sum()),
                0.0, 1.0);
    }

    private long countCombinations(int n, int m) {
        if (n < 0 || m < 0 || n < m) {
            throw new CreationException("Incorrect parameters while combinations " +
                "counting");
        }
        if (m * 2 < n) {
            m = n - m;
        }
        long result = 1;
        for (int i = m + 1; i <= n; i++) {
            result *= i;
        }
        for (int i = 1; i <= n - m; i++) {
            result /= i;
        }
        return result;
    }

    private double power(double x, int power) {
        double result = 1.0;
        for (int i = 0; i < power; i++) {
            result *= x;
        }
        return result;
    }
}
\end{verbatim}

\section*{\texttt{RotationCreator.java}}
\begin{verbatim}
package com.ran.engine.factories.interpolation.tools;

// Импорт классов
// ...

public class RotationCreator {

    private static final RotationCreator INSTANCE = new RotationCreator();

    public static RotationCreator getInstance() {
        return INSTANCE;
    }

    public DoubleMatrix createRotation(ThreeDoubleVector axis, double angle) {
        double n1 = axis.getX();
        double n2 = axis.getY();
        double n3 = axis.getZ();

        double sin = Math.sin(angle);
        double cos = Math.cos(angle);
        double vers = 1.0 - cos;

        double[][] matrix = new double[][] {
            {n1 * n1 + (1.0 - n1 * n1) * cos,
                    n1 * n2 * vers - n3 * sin,
                    n1 * n3 * vers + n2 * sin      },
            {n2 * n1 * vers + n3 * sin,
                    n2 * n2 + (1.0 - n2 * n2) * cos,
                    n2 * n3 * vers - n1 * sin      },
            {n3 * n1 * vers - n2 * sin,
                    n3 * n2 * vers + n1 * sin,
                    n3 * n3 + (1.0 - n3 * n3) * cos}
        };
        return new DoubleMatrix(matrix);
    }

    public Pair<ThreeDoubleVector, Double> getAxisAndAngleForRotation(
            DoubleMatrix rotation) {
        double r21Diff = rotation.get(1, 0) - rotation.get(0, 1);
        double r13Diff = rotation.get(0, 2) - rotation.get(2, 0);
        double r32Diff = rotation.get(2, 1) - rotation.get(1, 2);
        double sin = 0.5 * Math.sqrt(r21Diff * r21Diff + r13Diff * r13Diff +
                r32Diff * r32Diff);
        double cos = 0.5 * (rotation.get(0, 0) + rotation.get(1, 1) +
                rotation.get(2, 2) - 1.0);
        double phi = Math.atan2(sin, cos);
        ThreeDoubleVector axis;
        if (ArithmeticOperations.doubleEquals(phi, 0.0) ||
                ArithmeticOperations.doubleEquals(phi, Math.PI)) {
            axis = new ThreeDoubleVector(
                    Math.sqrt((rotation.get(0, 0) + 1.0) / 2.0),
                    Math.sqrt((rotation.get(1, 1) + 1.0) / 2.0),
                    Math.sqrt((rotation.get(2, 2) + 1.0) / 2.0)
            );
        } else {
            axis = new ThreeDoubleVector(
                    r32Diff / (2.0 * sin),
                    r13Diff / (2.0 * sin),
                    r21Diff / (2.0 * sin)
            );
        }
        return new Pair<>(axis, phi);
    }

    public DoubleMatrix createReversedRotationByRotation(DoubleMatrix rotation) {
        Pair<ThreeDoubleVector, Double> axisAndAngle =
                getAxisAndAngleForRotation(rotation);
        ThreeDoubleVector axis = axisAndAngle.getLeft();
        double angle = axisAndAngle.getRight();
        return createRotation(axis, -angle);
    }
}
\end{verbatim}

\section*{\texttt{CurvesDeformationCreator.java}}
\begin{verbatim}
package com.ran.engine.factories.interpolation.tools;

// Импорт классов
// ...

public class CurvesDeformationCreator {

    private static final CurvesDeformationCreator INSTANCE =
            new CurvesDeformationCreator();

    public static CurvesDeformationCreator getInstance() {
        return INSTANCE;
    }

    public <T extends AlgebraicObject<T>> DoubleFunction<T> deformCurves(
            DoubleFunction<T> firstCurve, DoubleFunction<T> secondCurve, int degree) {
        return deformCurves(firstCurve, secondCurve, degree,
                GroupMultiplicationOperationFactory.getMultiplicationOperation());
    }

    public <T extends AlgebraicObject<T>> DoubleFunction<T> deformCurves(
            DoubleFunction<T> firstCurve, DoubleFunction<T> secondCurve, int degree,
            BiFunction<DoubleFunction<T>, DoubleFunction<T>, DoubleFunction<T>>
                groupMultiplicationOperation) {
        if (!firstCurve.apply(0.0).equals(secondCurve.apply(0.0))) {
            throw new AlgebraicException("Start points of curves must coincide " +
                "for curves deformation");
        }
        DoubleFunction<SingleDouble> smoothingPolynom = PolynomsCreator.getInstance()
            .createSmoothingPolynom(degree);
        DoubleFunction<SingleDouble> tauMinus = new DoubleFunction<>(
                point -> new SingleDouble((1.0 - smoothingPolynom.apply(point)
                    .getValue()) * point), 0.0, 1.0);
        DoubleFunction<SingleDouble> tauPlus = new DoubleFunction<>(
                point -> new SingleDouble(smoothingPolynom.apply(point)
                    .getValue() * point), 0.0, 1.0);
        return groupMultiplicationOperation.apply(secondCurve.superposition(tauPlus),
            firstCurve.superposition(tauMinus));
    }

    public <T extends AlgebraicObject<T>> DoubleFunction<T> deformCurvesWithCommonEnd(
            DoubleFunction<T> firstCurve, DoubleFunction<T> secondCurve, int degree) {
        return deformCurvesWithCommonEnd(firstCurve, secondCurve, degree,
                GroupMultiplicationOperationFactory.getMultiplicationOperation());
    }

    public <T extends AlgebraicObject<T>> DoubleFunction<T> deformCurvesWithCommonEnd(
            DoubleFunction<T> firstCurve, DoubleFunction<T> secondCurve, int degree,
            BiFunction<DoubleFunction<T>, DoubleFunction<T>, DoubleFunction<T>>
                groupMultiplicationOperation) {
        if (!firstCurve.apply(0.0).equals(secondCurve.apply(0.0))) {
            throw new AlgebraicException("Start points of curves must coincide " +
                "for curves deformation");
        }
        DoubleFunction<SingleDouble> smoothingPolynom = PolynomsCreator.getInstance()
            .createSmoothingPolynom(degree);
        DoubleFunction<SingleDouble> tauMinus = new DoubleFunction<>(
                point -> new SingleDouble((1.0 - smoothingPolynom.apply(point)
                    .getValue()) * point), 0.0, 1.0);
        DoubleFunction<SingleDouble> tauMinusReversed = tauMinus.reversed();
        DoubleFunction<SingleDouble> tauPlus = new DoubleFunction<>(
                point -> new SingleDouble(smoothingPolynom.apply(point)
                    .getValue() * point), 0.0, 1.0);
        DoubleFunction<SingleDouble> tauPlusReversed = tauPlus.reversed();
        DoubleFunction<SingleDouble> tauPlusFixed = new DoubleFunction<SingleDouble>(
                point -> new SingleDouble(1.0 - tauPlusReversed.apply(point)
                    .getValue()), 0.0, 1.0);
        return groupMultiplicationOperation.apply(secondCurve.superposition(
            tauMinusReversed), firstCurve.superposition(tauPlusFixed));
    }
}
\end{verbatim}

\section*{\texttt{CurvesSmoothingCreator.java}}
\begin{verbatim}
package com.ran.engine.factories.interpolation.tools;

// Импорт классов
// ...

public class CurvesSmoothingCreator {

    private static final CurvesSmoothingCreator INSTANCE =
            new CurvesSmoothingCreator();

    public static CurvesSmoothingCreator getInstance() {
        return INSTANCE;
    }

    public <T extends AlgebraicObject<T>> DoubleFunction<T> smoothCurves(
            DoubleFunction<T> firstCurve, DoubleFunction<T> secondCurve, int degree) {
        return smoothCurves(firstCurve, secondCurve, degree,
                GroupMultiplicationOperationFactory.getMultiplicationOperation());
    }

    public <T extends AlgebraicObject<T>> DoubleFunction<T> smoothCurves(
            DoubleFunction<T> firstCurve, DoubleFunction<T> secondCurve, int degree,
            BiFunction<DoubleFunction<T>, DoubleFunction<T>, DoubleFunction<T>>
                groupMultiplicationOperation) {
        if (!firstCurve.apply(0.0).equals(secondCurve.apply(0.0))) {
            throw new AlgebraicException("Start points of curves must coincide " +
                "for curves deformation");
        }
        DoubleFunction<SingleDouble> smoothingPolynom = PolynomsCreator.getInstance()
            .createSmoothingPolynom(degree);
        DoubleFunction<SingleDouble> sigmaMinus = new DoubleFunction<>(
                point -> new SingleDouble((1.0 - smoothingPolynom.apply(point)
                    .getValue()) * (1.0 - point)), 0.0, 1.0);
        DoubleFunction<SingleDouble> sigmaPlus = new DoubleFunction<>(
                point -> new SingleDouble(smoothingPolynom.apply(point)
                    .getValue() * point), 0.0, 1.0);
        return groupMultiplicationOperation.apply(
            secondCurve.superposition(sigmaPlus),
            firstCurve.superposition(sigmaMinus));
    }
}
\end{verbatim}

\section*{\texttt{AbstractInterpolatedCurveCreator.java}}
\begin{verbatim}
package com.ran.engine.factories.interpolation.curvecreators;

// Импорт классов
// ...

public abstract class AbstractInterpolatedCurveCreator<
    I, O extends AlgebraicObject<O>,
    P extends InputParameters> implements InterpolatedCurveCreator<I, O, P> {

    protected void validateVerticesList(List<I> verticesList) {
        if (verticesList.size() < 3) {
            throw new InterpolationException("Interpolation requires " +
                "at least 3 vertices");
        }
    }

    protected DoubleFunction<TwoDoubleVector> buildFinalCurve(
            List<Double> timeMoments,
            List<DoubleFunction<TwoDoubleVector>> segments,
            int segmentsQuantity) {
        TimeMomentsUtil timeMomentsUtil = TimeMomentsUtil.getInstance();
        List<DoubleFunction<TwoDoubleVector>> curveSegments =
            new ArrayList<>(segmentsQuantity);
        for (int i = 0; i < segmentsQuantity; i++) {
            double startTime = timeMoments.get(i);
            double endTime = timeMoments.get(i + 1);
            DoubleFunction<TwoDoubleVector> currentSegment = segments.get(i);
            DoubleFunction<TwoDoubleVector> alignedCurveSegment =
                    currentSegment.superposition(timeMomentsUtil
                        .buildAligningFunction(startTime, endTime));
            curveSegments.add(alignedCurveSegment);
        }
        return DoubleMultifunction.makeMultifunction(curveSegments);
    }
}
\end{verbatim}
