\chapter{Интерполирование кривой на~двумерной сфере}

\section*{\texttt{AbstractSphereCurveCreator.java}}
\begin{verbatim}
package com.ran.engine.factories.interpolation.curvecreators;

// Импорт классов
// ...

public abstract class AbstractSphereCurveCreator extends AbstractInterpolatedCurveCreator<
        ThreeDoubleVector, ThreeDoubleVector, SimpleInputParameters> {

    @Override
    protected void validateVerticesList(List<ThreeDoubleVector> verticesList) {
        super.validateVerticesList(verticesList);
        double radius = verticesList.get(0).getNorm();
        if (verticesList.stream().anyMatch(vertice -> ArithmeticOperations.doubleNotEquals(
                vertice.getNorm(), radius))) {
            throw new InterpolationException("All vertices must belong to the same sphere");
        }
    }

    protected DoubleFunction<ThreeDoubleVector> buildFinalCurve(
            List<Double> timeMoments,
            List<ThreeDoubleVector> vertices,
            List<DoubleFunction<DoubleMatrix>> rotationsOnSegments,
            int segmentsQuantity) {
        TimeMomentsUtil timeMomentsUtil = TimeMomentsUtil.getInstance();
        List<DoubleFunction<ThreeDoubleVector>> curveSegments =
            new ArrayList<>(segmentsQuantity);
        for (int i = 0; i < segmentsQuantity; i++) {
            double startTime = timeMoments.get(i);
            double endTime = timeMoments.get(i + 1);
            DoubleFunction<DoubleMatrix> currentRotation = rotationsOnSegments.get(i);
            DoubleVector currentVertice = vertices.get(i).getDoubleVector();
            DoubleFunction<ThreeDoubleVector> curveSegmentWithoutAligning = new DoubleFunction<>(
                    point -> new ThreeDoubleVector(currentRotation.apply(point)
                        .multiply(currentVertice)), 0.0, 1.0
            );
            DoubleFunction<ThreeDoubleVector> alignedCurveSegment =
                    curveSegmentWithoutAligning.superposition(
                        timeMomentsUtil.buildAligningFunction(startTime, endTime));
            curveSegments.add(alignedCurveSegment);
        }
        return DoubleMultifunction.makeMultifunction(curveSegments);
    }
}
\end{verbatim}

\section*{\texttt{ArcsBuilder.java}}
\begin{verbatim}
package com.ran.engine.factories.interpolation.tools;

// Импорт классов
// ...

public class ArcsBuilder {

    private static final ArcsBuilder INSTANCE = new ArcsBuilder();

    public static ArcsBuilder getInstance() {
        return INSTANCE;
    }

    public Result buildArcsBetweenVerticesOnSphere(
            ThreeDoubleVector p1, ThreeDoubleVector p2, ThreeDoubleVector p3) {
        ThreeDoubleVector a = (p3.substract(p2)).multiply(p1.substract(p2));
        double aNorm = a.getNorm();
        if (ArithmeticOperations.doubleEquals(aNorm, 0.0)) {
            throw new AlgebraicException("Every three sequential vertices must not coincide");
        }

        double mixedProduction = p1.mixedMultiply(p2, p3);
        ThreeDoubleVector n = a.multiply(1.0 / aNorm);
        ThreeDoubleVector c = n.multiply(mixedProduction / aNorm);

        ThreeDoubleVector r1 = p1.substract(c);
        ThreeDoubleVector r2 = p2.substract(c);
        ThreeDoubleVector r3 = p3.substract(c);

        ThreeDoubleVector n1 = r1.multiply(r2);
        ThreeDoubleVector n2 = r2.multiply(r3);

        double n1Norm = n1.getNorm();
        double n2Norm = n2.getNorm();
        double s1 = r1.scalarMultiply(r2);
        double s2 = r2.scalarMultiply(r3);

        double firstAtan2 = Math.atan2(n1Norm, s1);
        double phi = -(n1.scalarMultiply(n) > 0 ? firstAtan2 : 2 * Math.PI - firstAtan2);

        double secondAtan2 = Math.atan2(n2Norm, s2);
        double psi = -(n2.scalarMultiply(n) > 0 ? secondAtan2 : 2 * Math.PI - secondAtan2);

        return new Result(
                new DoubleFunction<>(point -> RotationCreator.getInstance()
                    .createRotation(n, point * phi), 0.0, 1.0),
                new DoubleFunction<>(point -> RotationCreator.getInstance()
                    .createRotation(n, point * psi), 0.0, 1.0),
                phi, psi);
    }

    public static class Result {
        private final DoubleFunction<DoubleMatrix> firstRotation;
        private final DoubleFunction<DoubleMatrix> secondRotation;
        private final double firstAngle;
        private final double secondAngle;

        // Конструкторы, get- и set-методы
    }
}
\end{verbatim}

\section*{\texttt{SphereByPointsCurveCreator.java}}
\begin{verbatim}
package com.ran.engine.factories.interpolation.curvecreators;

// Импорт классов
// ...

public class SphereByPointsCurveCreator extends AbstractSphereCurveCreator {

    private static final SphereByPointsCurveCreator INSTANCE = new SphereByPointsCurveCreator();

    public static SphereByPointsCurveCreator getInstance() {
        return INSTANCE;
    }

    @Override
    public DoubleFunction<ThreeDoubleVector> interpolateCurve(List<ThreeDoubleVector> vertices,
                                                              SimpleInputParameters parameters,
                                                              int degree) {
        validateVerticesList(vertices);

        double t0 = parameters.getT0();
        double t1 = parameters.getT1();
        int k = vertices.size();

        CurvesDeformationCreator deformationCreator = CurvesDeformationCreator.getInstance();
        ArcsBuilder arcsBuilder = ArcsBuilder.getInstance();
        TimeMomentsUtil timeMomentsUtil = TimeMomentsUtil.getInstance();

        List<DoubleFunction<DoubleMatrix>> rotationsOnSegments = new ArrayList<>(k - 1);
        List<Pair<Double, Double>> rotationAngles = new ArrayList<>(k - 2);
        ArcsBuilder.Result currentArcsBuildingResult =
                arcsBuilder.buildArcsBetweenVerticesOnSphere(
                    vertices.get(0), vertices.get(1), vertices.get(2));
        rotationsOnSegments.add(currentArcsBuildingResult.getFirstRotation());
        rotationAngles.add(currentArcsBuildingResult.getAngles());

        for (int i = 1; i < k - 2; i++) {
            ArcsBuilder.Result nextArcsBuildingResult =
                    arcsBuilder.buildArcsBetweenVerticesOnSphere(
                        vertices.get(i), vertices.get(i + 1), vertices.get(i + 2));
            DoubleFunction<DoubleMatrix> deformedFunction = deformationCreator.deformCurves(
                    currentArcsBuildingResult.getSecondRotation(),
                    nextArcsBuildingResult.getFirstRotation(), degree);
            rotationsOnSegments.add(deformedFunction);
            rotationAngles.add(nextArcsBuildingResult.getAngles());
            currentArcsBuildingResult = nextArcsBuildingResult;
        }
        rotationsOnSegments.add(currentArcsBuildingResult.getSecondRotation());

        List<Double> timeMoments = timeMomentsUtil.countTimeMoments(rotationAngles, t0, t1, k);
        return buildFinalCurve(timeMoments, vertices, rotationsOnSegments, k - 1);
    }
}
\end{verbatim}

\section*{\texttt{BigArcsBuilder.java}}
\begin{verbatim}
package com.ran.engine.factories.interpolation.tools;

// Импорт классов
// ...

public class BigArcsBuilder {

    private static final BigArcsBuilder INSTANCE = new BigArcsBuilder();

    public static BigArcsBuilder getInstance() {
        return INSTANCE;
    }

    public Result buildBigArcBetweenVerticesOnSphere(
            ThreeDoubleVector p1, ThreeDoubleVector p2) {
        ThreeDoubleVector a = p1.multiply(p2);
        double aNorm = a.getNorm();
        if (ArithmeticOperations.doubleEquals(aNorm, 0.0)) {
            throw new AlgebraicException("Every two sequential vertices must not coincide");
        }

        ThreeDoubleVector n = a.multiply(1.0 / aNorm);
        double phi = -Math.atan(aNorm / p1.scalarMultiply(p2));

        return new BigArcsBuilder.Result(
                new DoubleFunction<>(point -> RotationCreator.getInstance()
                    .createRotation(n, point * phi), 0.0, 1.0), phi);
    }

    public static class Result {
        private final DoubleFunction<DoubleMatrix> rotation;
        private final double angle;

        // Конструкторы, get- и set-методы
    }
}
\end{verbatim}

\section*{\texttt{SphereBezierCurveCreator.java}}
\begin{verbatim}
package com.ran.engine.factories.interpolation.curvecreators;

// Импорт классов
// ...

public class SphereBezierCurveCreator extends AbstractSphereCurveCreator {

    private static final SphereBezierCurveCreator INSTANCE =
        new SphereBezierCurveCreator();

    public static SphereBezierCurveCreator getInstance() {
        return INSTANCE;
    }

    @Override
    public DoubleFunction<ThreeDoubleVector> interpolateCurve(
            List<ThreeDoubleVector> verticesList,
            SimpleInputParameters parameters, int degree) {
        validateVerticesList(verticesList);

        double t0 = parameters.getT0();
        double t1 = parameters.getT1();
        int k = verticesList.size();

        CurvesSmoothingCreator curvesSmoothingCreator = CurvesSmoothingCreator.getInstance();
        BigArcsBuilder bigArcsBuilder = BigArcsBuilder.getInstance();

        List<ThreeDoubleVector> arcsCenters = new ArrayList<>(k - 1);
        for (int i = 0; i < k - 1; i++) {
            BigArcsBuilder.Result bigArcsBuilderResult =
                bigArcsBuilder.buildBigArcBetweenVerticesOnSphere(
                    verticesList.get(i), verticesList.get(i + 1));
            arcsCenters.add(new ThreeDoubleVector(bigArcsBuilderResult.getRotation().apply(0.5)
                    .multiply(verticesList.get(i).getDoubleVector())));
        }

        List<DoubleFunction<DoubleMatrix>> halfRotationsForward = new ArrayList<>(k - 1);
        List<DoubleFunction<DoubleMatrix>> halfRotationsBack = new ArrayList<>(k - 1);
        for (int i = 0; i < k - 1; i++) {
            BigArcsBuilder.Result firstHalfBigArcsBuilderResult = bigArcsBuilder
                .buildBigArcBetweenVerticesOnSphere(
                    verticesList.get(i), arcsCenters.get(i));
            halfRotationsForward.add(firstHalfBigArcsBuilderResult.getRotation());
            BigArcsBuilder.Result secondHalfBigArcsBuilderResult = bigArcsBuilder
                .buildBigArcBetweenVerticesOnSphere(
                    verticesList.get(i + 1), arcsCenters.get(i));
            halfRotationsBack.add(secondHalfBigArcsBuilderResult.getRotation());
        }

        List<DoubleFunction<DoubleMatrix>> smoothedRotations = new ArrayList<>(k);
        smoothedRotations.add(halfRotationsForward.get(0));
        for (int i = 1; i < k - 1; i++) {
            smoothedRotations.add(curvesSmoothingCreator.smoothCurves(
                    halfRotationsBack.get(i - 1), halfRotationsForward.get(i), degree));
        }
        smoothedRotations.add(new DoubleFunction<>(
                point -> halfRotationsBack.get(k - 2).apply(1.0 - point),
                0.0, 1.0)
        );

        List<Double> timeMoments = new ArrayList<>(k + 1);
        double timeDelta = t1 - t0;
        for (int i = 0; i < k + 1; i++) {
            timeMoments.add(t0 + i * timeDelta);
        }
        return buildFinalCurve(timeMoments, verticesList, smoothedRotations, k);
    }
}
\end{verbatim}

\section*{\texttt{TangentBuilder.java}}
\begin{verbatim}
package com.ran.engine.factories.interpolation.tools;

// Импорт классов
// ...

public class TangentBuilder {

    private static final Logger LOG = LoggerFactory.getLogger(TangentBuilder.class);

    private static final TangentBuilder INSTANCE = new TangentBuilder();
    private static final DoubleMatrix Z_HALF_PI_ROTATION = RotationCreator.getInstance()
        .createRotation(ThreeDoubleVector.Z_ONE_THREE_DOUBLE_VECTOR, -Math.PI / 2.0);

    public static TangentBuilder getInstance() {
        return INSTANCE;
    }

    public Result buildTangent(ThreeDoubleVector point,
                               double tangentAngle,
                               Double forwardRotationAngle,
                               Double backRotationAngle) {
        LOG.trace("point = {}, tangentAngle = {}, forwardRotationAngle = {}, " +
            "backRotationAngle = {}", point, tangentAngle, forwardRotationAngle,
            backRotationAngle);
        ThreeDoubleVector a, b;
        if (ArithmeticOperations.doubleEquals(point.getX(), 0.0) &&
                ArithmeticOperations.doubleEquals(point.getY(), 0.0)) {
            a = ThreeDoubleVector.X_ONE_THREE_DOUBLE_VECTOR;
            if (point.getZ() > 0.0) {
                b = ThreeDoubleVector.MINUS_Y_ONE_THREE_DOUBLE_VECTOR;
            } else {
                b = ThreeDoubleVector.Y_ONE_THREE_DOUBLE_VECTOR;
            }
        } else {
            a = new ThreeDoubleVector(Z_HALF_PI_ROTATION.multiply(
                    new ThreeDoubleVector(point.getX(), point.getY(), 0.0)
                        .getDoubleVector())).normalized();
            b = point.multiply(a).normalized();
        }
        ThreeDoubleVector n = a.multiply(Math.sin(tangentAngle))
            .add(b.multiply(-Math.cos(tangentAngle)));
        LOG.trace("a = {}, b = {}, n = {}", a, b, n);

        DoubleFunction<DoubleMatrix> forwardRotation = null;
        DoubleFunction<DoubleMatrix> backRotation = null;

        if (forwardRotationAngle != null) {
            forwardRotation = new DoubleFunction<>(
                    u -> RotationCreator.getInstance().createRotation(
                        n, u * forwardRotationAngle),
                    0.0, 1.0);
        }
        if (backRotationAngle != null) {
            backRotation = new DoubleFunction<>(
                    u -> RotationCreator.getInstance().createRotation(n, -u * backRotationAngle),
                    0.0, 1.0);
        }
        return new Result(forwardRotation, backRotation, forwardRotationAngle,
            backRotationAngle);
    }

    public static class Result {
        private final DoubleFunction<DoubleMatrix> forwardRotation;
        private final DoubleFunction<DoubleMatrix> backRotation;
        private final Double forwardAngle, backAngle;

        // Конструкторы, get- и set-методы
    }
}
\end{verbatim}

\section*{\texttt{SphereByTangentAnglesCurveCreator.java}}
\begin{verbatim}
package com.ran.engine.factories.interpolation.curvecreators;

// Импорт классов
// ...

public class SphereByTangentAnglesCurveCreator extends AbstractInterpolatedCurveCreator<
        Pair<ThreeDoubleVector, Double>, ThreeDoubleVector, SimpleInputParameters> {

    private static final Logger LOG = LoggerFactory.getLogger(
        SphereByTangentAnglesCurveCreator.class);

    private static final SphereByTangentAnglesCurveCreator INSTANCE =
        new SphereByTangentAnglesCurveCreator();

    public static SphereByTangentAnglesCurveCreator getInstance() {
        return INSTANCE;
    }

    @Override
    public DoubleFunction<ThreeDoubleVector> interpolateCurve(
            List<Pair<ThreeDoubleVector, Double>> verticesWithTangentAnglesList,
            SimpleInputParameters parameters, int degree) {
        LOG.trace("verticesWithTangentAnglesList = {}, parameters = {}, degree = {}",
                verticesWithTangentAnglesList, parameters, degree);
        validateVerticesList(verticesWithTangentAnglesList);

        double t0 = parameters.getT0();
        double t1 = parameters.getT1();
        int k = verticesWithTangentAnglesList.size();
        LOG.trace("t0 = {}, t1 = {}, k = {}", t0, t1, k);

        CurvesDeformationCreator deformationCreator = CurvesDeformationCreator.getInstance();
        ArcsBuilder arcsBuilder = ArcsBuilder.getInstance();
        BigArcsBuilder bigArcsBuilder = BigArcsBuilder.getInstance();
        TangentBuilder tangentBuilder = TangentBuilder.getInstance();
        TimeMomentsUtil timeMomentsUtil = TimeMomentsUtil.getInstance();

        LOG.trace("Before calling ArcsBuilder");
        List<ArcsBuilder.Result> smallArcsResults = new ArrayList<>(k - 2);
        for (int i = 1; i < k - 1; i++) {
            smallArcsResults.add(arcsBuilder.buildArcsBetweenVerticesOnSphere(
                    verticesWithTangentAnglesList.get(i - 1).getLeft(),
                    verticesWithTangentAnglesList.get(i).getLeft(),
                    verticesWithTangentAnglesList.get(i + 1).getLeft()
            ));
        }
        LOG.trace("smallArcsResults = {}", smallArcsResults);

        LOG.trace("Before calling BigArcsBuilder");
        List<BigArcsBuilder.Result> bigArcsResults = new ArrayList<>(k - 1);
        for (int i = 0; i < k - 1; i++) {
            bigArcsResults.add(bigArcsBuilder.buildBigArcBetweenVerticesOnSphere(
                    verticesWithTangentAnglesList.get(i).getLeft(),
                    verticesWithTangentAnglesList.get(i + 1).getLeft()
            ));
        }
        LOG.trace("bigArcsResults = {}", bigArcsResults);

        LOG.trace("Before calling TangentBuilder");
        List<TangentBuilder.Result> tangentBuilderResults = new ArrayList<>(k);
        for (int i = 0; i < k; i++) {
            LOG.trace("Point #{} = {}", i, verticesWithTangentAnglesList.get(i));
            if (verticesWithTangentAnglesList.get(i).getRight() == null) {
                LOG.trace("Tangle angle is null");
                tangentBuilderResults.add(null);
            } else {
                LOG.trace("Processing angles");
                Double forwardAngle = null, backAngle = null;
                if (i + 1 < k) {
                    if (verticesWithTangentAnglesList.get(i + 1).getRight() != null ||
                            i + 1 == k - 1) {
                        forwardAngle = Math.abs(bigArcsResults.get(i).getAngle());
                    } else {
                        forwardAngle = Math.abs(smallArcsResults.get(i).getFirstAngle());
                    }
                }
                if (i - 1 >= 0) {
                    if (verticesWithTangentAnglesList.get(i - 1).getRight() != null ||
                            i - 1 == 0) {
                        backAngle = Math.abs(bigArcsResults.get(i - 1).getAngle());
                    } else {
                        backAngle = Math.abs(smallArcsResults.get(i - 2).getSecondAngle());
                    }
                }
                LOG.trace("forwardAngle = {}, backAngle = {}", forwardAngle, backAngle);
                tangentBuilderResults.add(tangentBuilder.buildTangent(
                        verticesWithTangentAnglesList.get(i).getLeft(),
                        verticesWithTangentAnglesList.get(i).getRight(),
                        forwardAngle,
                        backAngle
                ));
            }
        }

        List<Pair<Double, Double>> rotationAngles = new ArrayList<>(k - 2);
        for (int i = 0; i < k - 2; i++) {
            if (verticesWithTangentAnglesList.get(i + 1).getRight() == null) {
                rotationAngles.add(smallArcsResults.get(i).getAngles());
            } else {
                rotationAngles.add(tangentBuilderResults.get(i + 1).getAngles());
            }
        }
        LOG.trace("rotationAngles = {}", rotationAngles);

        LOG.trace("Before building rotationsOnSegments");
        List<DoubleFunction<DoubleMatrix>> rotationsOnSegments = new ArrayList<>(k - 1);
        for (int i = 0; i < k - 1; i++) {
            LOG.trace("Building rotation between points {} and {}: {} and {}",
                    i, i + 1, verticesWithTangentAnglesList.get(i),
                    verticesWithTangentAnglesList.get(i + 1));
            if (verticesWithTangentAnglesList.get(i).getRight() != null &&
                    verticesWithTangentAnglesList.get(i + 1).getRight() != null) {
                LOG.trace("Angles are set on both points");
                DoubleFunction<DoubleMatrix> firstDeformedCurve = deformationCreator
                        .deformCurves(
                        tangentBuilderResults.get(i).getForwardRotation(),
                        bigArcsResults.get(i).getRotation(), degree);
                DoubleFunction<DoubleMatrix> secondDeformedCurve = deformationCreator
                        .deformCurvesWithCommonEnd(
                        bigArcsResults.get(i).getRotation(),
                        tangentBuilderResults.get(i + 1).getBackRotation(), degree);
                rotationsOnSegments.add(deformationCreator.deformCurves(
                        firstDeformedCurve, secondDeformedCurve, degree));
            } else if (verticesWithTangentAnglesList.get(i).getRight() != null) {
                LOG.trace("Angle is set only on the first point");
                if (i == k - 2) {
                    rotationsOnSegments.add(deformationCreator.deformCurves(
                            tangentBuilderResults.get(i).getForwardRotation(),
                            bigArcsResults.get(i).getRotation(), degree));
                } else {
                    rotationsOnSegments.add(deformationCreator.deformCurves(
                            tangentBuilderResults.get(i).getForwardRotation(),
                            smallArcsResults.get(i).getFirstRotation(), degree));
                }
            } else if (verticesWithTangentAnglesList.get(i + 1).getRight() != null) {
                LOG.trace("Angle is set only on the second point");
                if (i == 0) {
                    rotationsOnSegments.add(deformationCreator.deformCurvesWithCommonEnd(
                            bigArcsResults.get(i).getRotation(),
                            tangentBuilderResults.get(i + 1).getBackRotation(), degree));
                } else {
                    rotationsOnSegments.add(deformationCreator.deformCurvesWithCommonEnd(
                            smallArcsResults.get(i - 1).getSecondRotation(),
                            tangentBuilderResults.get(i + 1).getBackRotation(), degree));
                }
            } else {
                LOG.trace("Angles are not set on both points");
                if (i == 0) {
                    rotationsOnSegments.add(smallArcsResults.get(i).getFirstRotation());
                } else if (i == k - 2) {
                    rotationsOnSegments.add(smallArcsResults.get(i - 1).getSecondRotation());
                } else {
                    rotationsOnSegments.add(deformationCreator.deformCurves(
                            smallArcsResults.get(i - 1).getSecondRotation(),
                            smallArcsResults.get(i).getFirstRotation(), degree));
                }
            }
        }

        List<Double> timeMoments = timeMomentsUtil.countTimeMoments(rotationAngles, t0, t1, k);
        LOG.trace("timeMoments = {}", timeMoments);

        LOG.trace("Before building curveSegments");
        List<DoubleFunction<ThreeDoubleVector>> curveSegments = new ArrayList<>(k - 1);
        for (int i = 0; i < k - 1; i++) {
            LOG.trace("Building curve between {} and {} points");
            double startTime = timeMoments.get(i);
            double endTime = timeMoments.get(i + 1);
            DoubleFunction<DoubleMatrix> currentRotation = rotationsOnSegments.get(i);
            DoubleVector currentVertice = verticesWithTangentAnglesList.get(i).getLeft()
                .getDoubleVector();
            LOG.trace("startTime = {}, endTime = {}, currentVertice = {}", startTime, endTime,
                currentVertice);
            DoubleFunction<ThreeDoubleVector> curveSegmentWithoutAligning = new DoubleFunction<>(
                    point -> new ThreeDoubleVector(currentRotation.apply(point)
                        .multiply(currentVertice)), 0.0, 1.0);
            DoubleFunction<ThreeDoubleVector> alignedCurveSegment =
                    curveSegmentWithoutAligning.superposition(timeMomentsUtil
                        .buildAligningFunction(startTime, endTime));
            curveSegments.add(alignedCurveSegment);
        }
        return DoubleMultifunction.makeMultifunction(curveSegments);
    }

    @Override
    protected void validateVerticesList(
            List<Pair<ThreeDoubleVector, Double>> verticesWithTangentAnglesList) {
        super.validateVerticesList(verticesWithTangentAnglesList);
        double radius = verticesWithTangentAnglesList.get(0).getLeft().getNorm();
        if (verticesWithTangentAnglesList.stream().anyMatch(
                verticeWithTangentAngle -> ArithmeticOperations.doubleNotEquals(
                    verticeWithTangentAngle.getLeft().getNorm(), radius))) {
            String message = "All vertices must belong to the same sphere";
            LOG.error(message);
            throw new InterpolationException(message);
        }
    }
}
\end{verbatim}
